\setcounter{equation}{0}
\setcounter{figure}{0}
\setcounter{table}{0}

\chapter{\label{chapter1} Introduction}

At the heart of science is the quest to further mankind's knowledge of the universe we live in. The Standard Model of particle physics is one of the greatest achievements in this effort, forming the basis of a description of the most fundamental building blocks of nature. However, despite its many successes verified in experimental physics, there are many indications it is not a complete theory. 

As particle physicists look inwards to smaller scales with higher energies, cosmologists look outwards into space. Cosmological experiments confirm that the matter of the observable universe accounts for only 4\% of the mass in the universe. Another type of matter, known as ``Dark Matter", accounts for 23\% and yet there is no particle in the existing Standard Model to account for this, indicating new physics.  Supersymmetry, one popular extension of the Standard Model predicts a new symmetry in which each known particle has an as-yet undiscovered partner. The lightest of these is stable and weakly interacting, and therefore could account for dark matter. 

Experimental particle physics pushes the frontier of energy ever upwards in order to probe the heart of matter to better resolution. The Large Hadron Collider is the first collider that can access physics on the TeV scale, where many hope the first indications of physics behind the Standard Model will lie. The Compact Muon Solenoid detector will collect data during these proton collisions for analysis in many areas of possible new physics. 

Motivated by Supersymmetry, this thesis details the search for signs of new physics consistent with a dark matter candidate particle. Events are required to have jets and missing energy where the candidate particle escapes the detector. The Standard Model theory is presented in Chapter~\ref{ch:theory} along with motivations for physics beyond, and a description of Supersymmetry. The data used is taken using the Compact Muon Solenoid Detector at the Large Hadron Collider, experimental descriptions of which are found in Chapter~\ref{ch:detector}, and the reconstruction performed prior to data release for the analysis users is described in Chapter~\ref{ch:objects}.


Chapter~\ref{ch:at} documents the design and verification of the novel background rejection variable \alt, using work undertaken by the author's analysis group in previous iterations of the analysis, and work on the leptonic definition undertaken by the author described in Section~\ref{sec:lalt}. The work presented in Chapter~\ref{ch:ra1} is documented in a public CMS Physics Analysis Summary~\cite{1fbnote} and published in Physical Review Letters~\cite{PRL1fb} in 2011. The work was undertaken by a small analysis group of which the author was a key active member singularly responsible for the $\mu$ control sample used for background prediction and in addition providing plots and yields for the signal selection. The work in Chapter~\ref{ch:ra4} represents an extension to the published analysis that is the sole work of the author, using the aforementioned leptonic definition of the \alt variable. 


