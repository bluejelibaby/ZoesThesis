\documentclass{article}
\setlength{\topmargin}{-.5in}
\setlength{\textheight}{9in}
\setlength{\textwidth}{6.25in}
\setlength{\oddsidemargin}{.125in}
\usepackage{amsmath}
\usepackage{graphics}
\usepackage{epsfig}
\begin{document}
Following the promising results of the $\alpha_{T}$ jet-balancing method previously described for the all-hadronic SUSY searches, it is a natural progression to look for extensions of this approach to the single-electron search, where a significant presence of QCD multi-jet backgrounds is expected.

The $\alpha_{T}$ variable is now redefined as an N-object system where the set of objects is 1 electron and N-1 jets. The new definition reproduces the kinematics of a di-jet system by contruction two pseudo-jets, which balance one another in $H_{T}$. Pseudo-jets are formed from the combination of the objects by minimising the variable $\Delta H_{T} = |H_{T,1} - H_{T,2}|$, where the definition of the $\alpha_{T}$ variable is now written as:

\begin{equation}
\alpha_{T} = \frac{1}{2} \frac{H_{T} - \Delta H_{T}}{M_{T}} =  \frac{1}{2} \frac{H_{T} - \Delta H_{T}}{\sqrt{H_{T}^{2}-MH_{T}^{2}}}
\end{equation}
The basic selection for single electron events in the leptonic $\alpha_{T}$ approach has the follwoing requirements:
\begin{itemize}


\item Exactly one electron passing requirements as follows:
\begin{itemize}
\item $p_{T} >$ 20 GeV
\item $|\eta| <2.4$
\item Passes the Cut Based ID formed by simple, yet robust, variables (these are the $H/E$, the super-cluster (SC) - track matching variables $\Delta \phi$,$\Delta \eta$, and shower shape variable $\sigma_{i\eta i\eta}$, Tracker Isolation, ECAL Isolation and HCAL Isolation\footnote{The cut based ID selection cuts are taken such so that they correspond to an 80\% efficiency in the $W\rightarrow e\nu$ analysis.}). 
\end{itemize}
\item The event is vetoed if there are any muons passing the following requirements:
\begin{itemize}
\item $p_{T} >$ 15 GeV
\item $|\eta| <2.1$
\item Passes ID requirement: IsGlobalMuon
\end{itemize}
\item A jet is rejected if it is found close to a tight and isolated electron within $\Delta R$=0.5.
 The jet selection is as follows:
\begin{itemize}
\item $p_{T} >$ 20 GeV
\item $|\eta| <5$
\item EMF $<$0.9
\end{itemize}

\end{itemize}

The analysis uses data samples for the QCD multi-jet background processes produced with the full simulation production for Physics at 7TeV with CMS. The Monte Carlo datasets used are detailed in Table \ref{tab:datasets}. The available luminosity is shown, although the plots are normallised to 1pb$^{-1}$. We use the EMenriched and BCtoE samples which are designed to be enriched in electrons, in order to enhance statistics. In addition, as these samples only cover the region 20$<\hat{p_{T}}<$170 GeV/c, we include the inclusive QCD Jets sample for the region $\hat{p_{T}}>$170 GeV/c.

\begin{table}[h!]
\begin{center}
\begin{tabular}{|c|c|c|c|}
\hline
Data Set & N events & $\sigma$ (pb) & Equivalent luminosity (pb$^{-1}$)\\
\hline
QCD BCtoE [20$<\hat{p_{T}}<$30]& 432380  & 108330 & 3.99\\
QCD BCtoE [30$<\hat{p_{T}}<$80] & 840100 & 138762 & 6.05\\
QCD BCtoE [80$<\hat{p_{T}}<$170 & 682720 & 9422.4 &72.4\\
QCD EM Enriched [20$<\hat{p_{T}}<$30] & 6169999 & 1719150 & 3.59\\
QCD EM Enriched [30$<\hat{p_{T}}<$80] & 9054696 & 3498700 & 2.59\\
QCD EM Enriched [80$<\hat{p_{T}}<$170] & 2492814 & 134088 & 18.59\\
QCDJets $\hat{p_{T}}>$170 & 3132800 & 25470 & 122.99\\


\hline

\end{tabular}
\end{center}
\caption{\textit{The Monte Carlo datasets used to investigate the Delta ID Inversion method in QCD backgrounds. The available Lumniosity is shown, although plots produced are normalised to 1pb$^{-1}$ for the purpose of understanding near-reach of CMS.}}
\label{tab:datasets}
\end{table}

This session is dedicated to a first approach of commissioning the alphaT observable and study its behavior in pure fake electron events. It is therefore desirable to collect a suitable control sample which will be dominated by fake electrons whereas eliminate sources of prompt electron events (like W events).
. One way to obtain such a sample is using the anti-selection method on electron ID variables which are less correlated with the missing transverse energy. In this section, we investigate the possibility of inverting the $\Delta \eta$(trk-SC) and $\Delta \phi$ (trk-SC) id cuts in the electron selection. The anti-selected events in this method are collected with electrons that pass all selection id criteria except the $\Delta \phi$ (trk-SC) and $\Delta \eta$(trk-SC) ones. 

In order to establish the anti-selection method above, we compare next the perfomance of the leptonic alphaT as obtained from the control sample and the actual QCD events passing the electron criteria defined in the ``signal'' region.

As mentioned previously, SUSY events are expected to have high H$_{T}$. Therefore it is critical to understand the behaviour in leptonic AlphaT as a function of HT. The plots in \ref{fig:AlphaTbyHT} show the normalised shape of distributions, both before HT cut and the evolution with HT. The selected and anti-selected distributions agree very well.
\begin{figure}[h!]

\includegraphics[width=50mm]{alphat_1.png}
\includegraphics[width=50mm]{alphat_2.png}
\includegraphics[width=50mm]{alphat_3.png}
\includegraphics[width=50mm]{alphat_4.png}
\hspace*{3mm}
\includegraphics[width=50mm]{alphat_5.png}
\hspace*{2mm}
\includegraphics[width=50mm]{alphat_6.png}
\caption{\textit{The $\alpha_{T}$ distributions for selected (red) and anti-selected events (blue) from inversion of the $\Delta \phi$ and $\Delta \eta$ ID Cutts, shown without HT cut (Top Left) and with progressive HT cuts (left-right, top-bottom). These distributions are normalised to unity for shape comparison. There is good agreement between the selected and anti-selected regardless of HT requirement, and the high $\alpha_{T}$ tails reduce as expected when moving to higher HT cuts.}}
\label{fig:AlphaTbyHT}
\end{figure}

In order to demonstrate the power of HT in $\alpha_{T}$ tail-reduction, we introduce the variable $R_{\alpha_T}$ which is defined as the ratio of the number of events passing the $\alpha_T$ cut over the number of events failing it:
\begin{equation}
R_{\alpha T} = \frac{N(\alpha_{T}>0.55)}{N(\alpha_{T}<0.55)}
\end{equation}
The ``default'' cut value here is the value prompted from the all-hadronic analysis, 0.55. Figure \ref{fig:AlphaT_Ratio} shows a plot of $R_{\alpha_T}$ as a function of the $H_{T}$ cut applied. As the $H_{T}$ requirement increases, $R_{\alpha_T}$ decreases in an exponential manner. The selected and anti-selected events from the electron id inversion method are in good agreement.

\begin{figure}[h!]
\begin{center}
\includegraphics[width=100mm]{AlphaTRatio.png}
\end{center}
\caption{\textit{The $R_{\alpha_T}$ versus the $H_{T}$ cut applied for the QCD multi-jet background, shown for both selected and anti-selected events in the Delta ID Inversion method.}}
\label{fig:AlphaT_Ratio}
\end{figure}

In order to verify that the tail is not reduced proportionally to the peak, an additional set of ratios may be defined. Defining a reference istribution for low $H_{T}$ region [0,100], we can investigate distributions in higher HT bins ([100,150],[150,200],[200,250]) by dividing them by the reference distribution. These plots are shown in Figure \ref{fig:HTratios}. As the region of $H_{T}$ is increased, the difference in ratio between the peak and the tail becomes more pronounced. 

\begin{figure}[h!]

\includegraphics[width=50mm]{ratio1.png}
\includegraphics[width=50mm]{ratio2.png}
\includegraphics[width=50mm]{ratio3.png}

\caption{\textit{The ratio of $\alpha_{T}$ distributions in high $H_{T}$ regions(Left:[100,150] GeV ,Middle:[150,200] GeV ,Right:[200,250] GeV) to the $\alpha_{T}$ distribution in a low $H_{T}$ region([0,100] GeV). Both selected (red) and anti-selected events (blue) from inversion of the $\Delta \phi$ and $\Delta \eta$ ID Cuts are shown.}}
\label{fig:HTratios}
\end{figure}


\end{document}
