\section{Introduction}
\label{sec:intro}

We present a general-purpose search for supersymmetry (SUSY) in events containing one (and only one) lepton ($e$ or $\mu$, but not $\tau$)), multiple jets and large missing transverse energy. 

In terms of hadronic requirements, the single-lepton channels maintain the inclusive nature of the all-hadronic searches (i.e. jets plus missing energy), while the requirement of a charged lepton leads, in principle, to a cleaner signature, albeit at the usual reduction of signal events. Past studies have shown that single-lepton searches, with a significant suppression of the QCD N-jet background and despite the reduced production rate, can reach a sensitivity which is similar that of the all-hadronic channel. Clearly, the Electroweak (EWK) (W/Z + jets, di-bosons etc), and $t\bar{t}$ + jets backgrounds are now more important; it is hoped that these backgrounds will be measured, understood and thus controlled to a greater extent than the all-hadronic QCD ones.

This analysis utilizes the kinematic $\alpha_{T}$ jet-balancing method which has been recently developed within CMS, as a generic approach to discover New Physics (most favorably supersymmetry) in the single-lepton plus missing energy channel. We study the extension of the $\alpha_{T}$ method from the all-hadronic channels, with main motivation the reduction of the QCD background and the reliable control over severe jet mismeasurements. This reduction and control of the QCD background is particularly important for the one-lepton analysis when the lepton transverse momentum is lowered to values as low as 5~GeV. This, in turn, would allow to cover more of the important soft lepton parameter space predicted by many SUSY models.

The present analysis is studied in the context of the early physics data at the LHC and demonstrates a promising way of establishing a deviation of New Physics from the Standard Model expectations. The method relies entirely on kinematic properties of a SUSY signal and therefore provides a model-independent way for SUSY discovery.

\section{Analysis Framework}
\label{sec:framework}

The analysis uses data samples for the signal and background processes produced with the summer08/Fall08 full simulation production for Physics at 10~TeV \cite{data}, with CMS. The Standard Model background processes considered are listed below:
\begin{itemize}
\item QCD as well as $b\bar{b} + \textrm{N - jet}$ processes, in complete bins of the $\hat{p_{T}}$ ([100, 250], [250, 500], [500, 1000], [1000, inf]), were produced with the event generator MadGraph \cite{mad}. MadGraph is a matrix-element event generator, where higher order effects, like the emission of extra ISR gluons, are included in the matrix-element calculation\footnote{This is opposed to parton-shower based generators, like PYTHIA, where the emission of extra jets, other than the ones emitted by the $2 \rightarrow 2$ hard scattering process, is simulated by the parton-shower model.}. The total QCD bacgkround consists of approximately 20~M events, whereas the $b\bar{b}$ sample is $\approx 10$~M. The main QCD sources which stem as a background to single-lepton SUSY searches, originate from the presence of heavy-flavor muons, fake electrons and to a minor extent from the presence of heavy-flavor electrons, when these leptons survive the quality as well as the isolation requirements.
\item $t \bar{t}$ events produced with MadGraph. The single-lepton background arise from the semi-leptonic top decays, and to less extent from the di-leptonic top decays with one lepton missed due to acceptance or other quality criteria.
\item $W + $ jets events produced with MadGraph. $W (\rightarrow e \nu / \mu \nu)$ events are expected to play an important role especially when the two-jet bin is included, due to its high cross-section.
\item $Z + $ jets events produced with MadGraph. Z to di-lepton ($ee / \mu\mu$) background events appear when one lepton is lost due to acceptance or failing the quality/isolation criteria.
\end{itemize}

The signal samples of the analysis are taken among the Low Mass mSUGRA benchmark points (LMx , x=1,..11 benchmarks) of CMS, generated with PYTHIA 6, and were used to estimate signal yields for SUSY scenarios relevant to early SUSY searches (LM0 and LM1 test points). More details on these SUSY models can be found here \cite{lmx}. \\

The coding structure used for this analysis has been developed in CMSSW\_2\_2\_X releases on top of the SusyAnalysis software package \cite{susypat1}, which is itself an extension of the Physics Analysis Toolkit (PAT) \cite{susypat2}. A detailed description of the code can be found here \cite{susypat}. The PAT provides post-processing of reconstructed event data, in order to eliminate the information and condense the number of physics objects in an event for simplified physics analysis purposes. The framework comprises three layers. The initial layer reprocesses RECO or AOD data with the aim to refine the reconstructed object collections (remove duplicates for example).

The second layer formulates the cleaned data into simple object collections, such as PAT::Jet, PAT::Electron, PAT::Muon etc which are sorted in uncorrected transverse energy.  At this stage, the data are available for use in analysis. A third layer may optionally used as well, providing utilities such as cross cleaning between various object collections. The purpose of the cross-cleaning module is to eliminate overlaps between physics objects which share energy deposits, such as jets and electrons, and correct their energy appropriately. The output of the above PAT processing steps, is a ROOT ntuple which is further analysed with private code as described here \cite{ICNT}.