\section{Summary}
\label{summary}

The present analysis has been studied in the context of the single-lepton SUSY searches with CMS, proposing an alternative method to the traditional $ME_{T}$-based analyses, with the application of the $\alpha_{T}$ jet-balancing method. Following the promising results of the all hadronic SUSY analysis, the $\alpha_{T}$ method has been shown to provide equivalent perfomance in terms of robustness and reliability to control the QCD background  in the single-lepton mode SUSY searches. This is especially important when the lepton transverse momentum is lowered to values as low as 5~GeV, which results in increasing the QCD background by a huge factor. However, this offers the additional advantage to cover more of the SUSY parameter space where relatively soft leptons are predicted.

The analysis has used the signal low mass SUSY benchmark points, LM0 and LM1, assuming an integrated luminosity of 100 $\textrm{pb}^{-1}$ at a centre of mass  energy of 10 TeV. A model independent method has been developed to establish a deviation of SUSY from the SM expectations, by utilizing kinematic-only observables (e.g. jet $\eta$ versus $H_{T}$ method). The results have shown that a significant deviation can be established with the very early data at LHC. For the sake of completeness, the results obtained form this purely kinematic method, have been compared with the official recommendation of the CMS SUSY group, - RA4 approach-, which relies on the measurement of a missing energy from the calorimeters of the detector. It has been shown that the $\alpha_{T}$ selection performs similarly with RA4, in terms of significance and signal-to-background yields. Furthermore, the $\alpha_{T}$ method exposed in various systematic variations of the jet energies, showing a remarkable stability in performance even after severe jet mismeasurements with the CMS detector.