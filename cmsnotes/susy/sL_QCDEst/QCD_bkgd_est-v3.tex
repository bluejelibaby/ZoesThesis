\section{QCD Background Estimation Methods}

\label{sec:QCD_bkgd_est}

In the context of a search for SUSY using a signature containing an electron, there are three sources of ``background electrons'', namely ``non-prompt'' electrons:
\begin{enumerate}
\item Jets which are either mismeasured or have very atypical hadronization, yielding ``electrons'' which pass the basic identification and selection requirements.  We refer to these as ``Jet electrons'' (Jet-e) in what follows.
\item Photon conversions in the tracker material.  We refer to these as ``Conversion electrons'' (Conv-e) in what follows.
\item Electrons from semileptonic decays of $c$ and $b-$ flavored hadrons.  We refer to these as ``Heavy Flavor electrons'' (HF-e) in what follows.
\end{enumerate}

There are numerous methods for estimating both the amount of background and the shape of this background as a function of various variables (the transverse momentum, pseudorapidity, and in some cases more complicated topological variables) that remains in a data sample after a specific set of selection cuts.  All these methods employ ``control samples'' which are selected via appropriate alternate requirements which do not affect the shape of the variable in question (e.g. the transverse momentum of the lepton).  As an example, we mention the well-known ``Isolation inversion'' method, where one uses events failing the isolation requirement and an extrapolation into the ``selection region'', i.e. where events pass the isolation requirement, to estimate the amount of background in the latter~\cite{cornel}.  A crucial issue in most methods is the demonstration that the extrapolation from the ``control region'' to the ``selection region'' is correct or, at least, its deficiencies are small and can be estimated reliably.  There are two elements that enter this extrapolation from a control region to a selection region in the case of electrons:
\begin{itemize}
\item The actual knowledge of the full dependence of any single source of background electrons on the variable in question.  As an example, concentrating on only the Jet-e background, there remains the issue of how to obtain reliably the full shape of the isolation of the Jet-e background which passes all selection cuts from some other ``control sample'' which passes a slightly different set of cuts.
\item The presence of the three different sources of background implies that each of these backgrounds may well exhibit a different dependence on the variable in question.  As an example, the isolation distribution from Jet-e and Conv-e need not be (and is in fact not) the same.  This implies that it may be necessary to determine explicitly the amount of each background in the selection region, and to then form the total expected background as the sum of the three individual components.
\end{itemize}
For the sake of concreteness, in what follows we will refer to the ``Isolation variable'', even though the discussion applies equally well to essentially most variables.

Of the two above issues, the first is usually tackled initially using Monte Carlo: one uses generator-level information to obtain the Isolation distribution for any specific background (e.g. Jet-e) and then compares this shape with that extracted from the ``control sample''.  Assuming that this comparison is favorable, i.e. that the shape from the control sample can adequately describe the shape in the selection sample, the next and final step is to compare this shape with the one extracted from data using the ``control selection''. 

Tackling the second issue is more complicated though. First, the presence of three sources of background (instead of two) implies more degrees of freedom in the fitting (and in general in describing) any particular variable -- e.g. the Isolation variable. Second, the relative amount of each source needs to be determined for the final selection sample.

In this section, we concentrate on the following three key observations to address all of the above:
\begin{enumerate}
\item A good method for identifying a control sample for the Jet-e background is the reversal of the matching cuts $\Delta\eta$ and $\Delta\phi$.  These two variables exhibit near independence from other selection variables, especially the isolation variable
\item The isolation distributions of the Jet-e and HF-e backgrounds seem to be similar and to be collectively described well by the same control sample (via the matching-cut reversal).
\item A fairly pure sample of Conv-e background can be identified using the existing conversion-identification tools. This sample can then be used to extract a template for other variables in question, e.g. the Isolation distribution or the $\alpha_T$ distribution.
\end{enumerate}
In what follows, we exploit these three observations to obtain the shapes of the total background in both the isolation and $\alpha_T$ distributions.  Moreover, a fit to the isolation distribution using different shapes (templates) for the Jet-e and Conv-e backgrounds yields an estimate of the absolute number of background events in the selection region.

\subsection{Modeling the electron backgrounds}
\label{sec:modeling}

In what follows, we will consider more closely two different definitions of ``Isolation'', namely ``Calo Isolation'', which is defined using only the sum of the energies in the ECAL and HCAL, and ``Combined Isolation'' which includes also the transverse momenta of charged-particles tracks around the electron.  Since the search for SUSY may necessitate using low-$P_T$ leptons, we investigate the behavior for two different thresholds of 10 and 20 GeV on the electron $P_T$.

The isolation distribution using only calorimeter isolation is displayed in Figure~\ref{fig:caloIso_MC}, whereas the combined (relative) isolation is shown in Figure~\ref{fig:combIso_MC}.  It can be seen that two backgrounds, namely the Jet-e and JF-e have distributions which are fairly similar for both the Calo-Iso and Comb-Iso.
\begin{figure}[htb!]
\centering
\includegraphics[scale=0.32]{Plots/caloIso_pt10_MC.png}
\includegraphics[scale=0.32]{Plots/caloIso_pt20_MC.png}
\caption{\textit{The Calorimeter Isolation distribution from Monte Carlo simulation (QCD pythia $\hat{p}_{T} > 80$).  On the left for a threshold of 10 GeV on the electron and the right for a 20 GeV threshold.  The three sources of background, namely Jet-e, HF-e and Conv-e are shown separately, along with the sum of the three.}}\label{fig:caloIso_MC}
\end{figure}

\begin{figure}[h!]
\centering
\includegraphics[scale=0.32]{Plots/combIso_pt10_MC.png}
\includegraphics[scale=0.32]{Plots/combIso_pt20_MC.png}
\caption{\textit{The Combined Isolation distribution from Monte Carlo simulation  (QCD pythia $\hat{p}_{T} > 80$).  On the left for a threshold of 10 GeV on the electron and the right for a 20 GeV threshold.  The three sources of background, namely Jet-e, HF-e and Conv-e are shown separately, along with the sum of the three.}}\label{fig:combIso_MC}
\end{figure}

%\clearpage

\subsubsection{Isolation Distribution for the Jet-e and HF-e backgrounds}

The first step is to define a control sample for the combination of Jet-e and HF-e and to see how well it can describe the Isolation distributions (both CaloIso and CombIso) in the selection region.  As stated previously, this is done by inverting the $\Delta \eta$(trk-SC) and $\Delta \phi$ (trk-SC) id cuts in the electron selection. The selected events in this method pass the pre-selection described in Section~\ref{sec:evtsel}, while the anti-selected are those events which pass the selection with an electron that passes all selection criteria {\it except} the $\Delta \phi$ (trk-SC) and $\Delta \eta$(trk-SC) ones.  The resulting distributions are shown in Figure~\ref{fig:caloIso_fakes} for the CaloIso and in Figure~\ref{fig:combIso_fakes} for the CombIso.

\begin{figure}[ht!]
\centering
\includegraphics[scale=0.32]{Plots/caloIso_pt10_fakes.png}
\includegraphics[scale=0.32]{Plots/caloIso_pt20_fakes.png}
\caption{\textit{The Calorimeter Isolation distribution from Monte Carlo simulation of the combined Jet-e+HF-e background.   On the left for a threshold of 10 GeV on the electron and the right for a 20 GeV threshold.  The solid blue line is the total Jet-e+HF-e background in the selection region, whereas the dashed line is the distribution from the control sample, defined via the anti-selection on the matching cuts.  The solid black line is the result of re-weighting the control sample for the jet spectra -- as described in the text.}}\label{fig:caloIso_fakes}
\end{figure}

\begin{figure}[h!]
\centering
\includegraphics[scale=0.32]{Plots/combIso_pt10_fakes.png}
\includegraphics[scale=0.32]{Plots/combIso_pt20_fakes.png}
\caption{\textit{Same as Figure~\ref{fig:caloIso_fakes} only this time for the Combined Isolation distribution.}}\label{fig:combIso_fakes}
\end{figure}

It can be seen that the shape of the Combined Isolation distribution from the Control Region is quite similar to that of the distribution from the Selection Region -- a very encouraging result which indicates that the anti-selection of the matching cuts yields a good method for modeling non-conversion electrons. A closer look into various properties of these events, however, yields a slight difference in the Calorimeter Isolation distributions.  As can be seen in Figure~\ref{fig:caloIso_fakes}, for small values of the CaloIso variable the two distributions, i.e. from the Control and Selection regions have a small difference (around CaloIso$\approx 0.1-0.4$).  This is more easily seen in Figure~\ref{fig:ratio_control_fakesHF} where the ratio of the Calorimeter Isolation distributions from the Selection and Control regions is displayed. 

\begin{figure}[h!]
\centering
\includegraphics[scale=0.32]{Plots/ratio_control_fakesHF.png}
\includegraphics[scale=0.32]{Plots/ratio_re-control_fakesHF.png}
\caption{\textit{The ratio of the Calorimeter Isolation distributions from the Selection Region and the Control Region.  Left plot: the ratio of the two ``raw'' distributions (the dashed curves) seen in Figure~\ref{fig:caloIso_fakes}.  Right plot: {\it after} the re-weighting correction, i.e. the ratio of the two black lines in Figure~\ref{fig:caloIso_fakes}.}}\label{fig:ratio_control_fakesHF}
\end{figure}

We have investigated possible sources of this difference, from the surrounding objects, and in particular any jet that (possibly) accompanies the electron candidate.  There are small differences in the shapes of the associated jet $P_T$ spectrum as well as from the distance in $\eta-\phi$ space ($\Delta R$) between the jet and the electron.  These can be seen in Figure~\ref{fig:Pt-DR-comparison}.

\begin{figure}[h!]
\centering
\includegraphics[scale=0.32]{Plots/jetEt_dependence.png}
\includegraphics[scale=0.32]{Plots/dR_dependence.png}
\caption{\textit{Distributions of the transverse momentum ($P_T$) and distance to the electron $\Delta R$) of the nearest jet.}}\label{fig:Pt-DR-comparison}
\end{figure}

Presumably, these differences arise from a small correlation between the matching variables and the density of the overall hadronic energy surrounding or near the electron candidate.  Since the difference is small, we have attempted to correct the ``predicted'' shape, i.e. the shape from the Control Region, by applying a weight which depends on the associated Jet $P_T$ and the $\Delta R$ between the jet and the electron.  The solid blue line in \ref{fig:caloIso_fakes} is the result of this re-weighting.  It can be seen that the corrected Calorimeter Isolation distribution from the Control Region now describes the distribution from the Selection Region quite well.  This is more readily seen in Figure~\ref{fig:ratio_control_fakesHF} (right).

We have also applied the same correction to the Combined Isolation distribution, which did not exhibit a visible difference between the Control and Selection Regions, to ensure that the correction did not adversely affect this original agreement.  As can be seen in Figure~\ref{fig:combIso_fakes} the corrected distribution still describes the Selection Region very well.

\subsubsection{Isolation distribution from the Conv-e background}

Electrons from photon conversions are identified using the standard criteria of the Conversion Finder tools \cite{conv}. A suitable Control sample for modeling the Conv-e component in the Isolation distribution can be obtained by electrons that pass the Conversion tools (i.e. using an anti-veto on the Conversion rejection requirements). There are two algorithms to identify electrons from converted photons:

\begin{itemize} 
\item \textit{Missing expected hits}: the algorithm asks that there be $> 0$ expected layers with a missing hit before the first valid hit on the electron's track. The number of missing expected hits in front of the innermost valid hit is available via the electron's gsfTrack Hit Pattern.
\item \textit{Partner track finding}: the algorithm looks for the electron's partner track from a converted photon in the generalTrack collection. The track is identified as a conversion partner if: 
\begin{itemize}
\item the track has opposite charge to the electron track.
\item Approximately the same $\delta \cot(\theta)$ , in this case: $|\delta \cot(\theta)| < 0.02$.
\item small distance (dist) in the $r-\phi$ plane, in this case: $|\text{dist}| < 0.02$. 
\end{itemize}
\end{itemize}

The Conv-e component of the Isolation in the Selection region is formed using only electrons which match a generated photon at MC generator (``MC-truth'') level. The shape of this component is compared with the one obtained from the Conv-e control region as described previously.  The resulting distributions are shown in Figure~\ref{fig:caloIso_conv} for the CaloIso and in Figure~\ref{fig:combIso_conv} for the CombIso.  It can be seen that the shape of both distributions in the Selection Region is described well by the Control Region.

\begin{figure}[h!]
\centering
\includegraphics[scale=0.32]{Plots/caloIso_pt10_conv.png}
\includegraphics[scale=0.32]{Plots/caloIso_pt20_conv.png}
\caption{\textit{The Calorimeter Isolation distribution from Monte Carlo simulation of electrons from conversions.   On the left for a threshold of 10 GeV on the electron and the right for a 20 GeV threshold.  The solid red line is the Conv-e background in the selection region, whereas the dashed line is the distribution from the control sample, defined via the conversion identification requirements described in the text.}}\label{fig:caloIso_conv}
\end{figure}

\begin{figure}[h!]
\centering
\includegraphics[scale=0.32]{Plots/combIso_pt10_conv.png}
\includegraphics[scale=0.32]{Plots/combIso_pt20_conv.png}
\caption{\textit{Same as Figure~\ref{fig:caloIso_conv} only this time for the Combined Isolation distribution.}}\label{fig:combIso_conv}
\end{figure}

\subsubsection{Describing the Isolation distribution in the Selection Region}

Having demonstrated that two independent Control Samples can yield good descriptions of the Isolation distribution in the Selection Region for each background (i.e the combined Jet-e+HF-e and the Conv-e) we next attempt to describe the full Isolation distribution in the Selection Region as a sum of two components, with the template of each component extracted as described above: the combined Jet-e and HF-e background is described from the corrected (re-weighted) anti-selected (for matching) sample, whereas the Conv-e background is described from the sample passing the conversion-identification criteria.  We thus fit the total Isolation distribution using these two components, leaving the relative normalization of the two as a free fit parameter.  The result is shown in Figure~\ref{fig:caloIso_fit} for the calorimeter isolation and shows a very good description of the total background distribution.  It can also be observed that the conversion component becomes more relevant at high $P_T(e)$.  The corresponding distributions for the Combined Isolation variable are described equally well (see Figure~\ref{fig:combIso_fit}).
%  In the interest of saving space, and since previously it is the Calorimeter Isolation distribution which exhibited some differences, in what follows we will concentrate only on the Calorimeter Isolation (even though the corresponding Combined Isolation was always checked and found to be in excellent agreement).

\begin{figure}[h!]
\centering
\includegraphics[scale=0.32]{Plots/caloIso_pt10_fit.png}
\includegraphics[scale=0.32]{Plots/caloIso_pt20_fit.png}
\caption{\textit{The Calorimeter Isolation distribution from Monte Carlo simulation of background electrons. On the left for a threshold of 10 GeV on the electron and the right for a 20 GeV threshold.  The dashed lines are the two background components as extracted from the two Control Samples, whereas the sum of the ``predicted'' background is the solid line.}}\label{fig:caloIso_fit}
\end{figure}

\begin{figure}[h!]
\centering
\includegraphics[scale=0.32]{Plots/combIso_pt10_fit.png}
\includegraphics[scale=0.32]{Plots/combIso_pt20_fit.png}
\caption{\textit{The combined Isolation distribution from Monte Carlo simulation of background electrons. On the left for a threshold of 10 GeV on the electron and the right for a 20 GeV threshold.  The dashed lines are the two background components as extracted from the two Control Samples, whereas the sum of the ``predicted'' background is the solid line.}}\label{fig:combIso_fit}
\end{figure}

\subsubsection{Describing the Isolation distribution in the presence of prompt electrons}
\label{mc_wcontamin}

In the data, the Isolation distribution in the Selection region will also be populated by prompt electrons sources -in addition to QCD sources - with  $W \rightarrow e\nu$ being the most prominent one (and electrons from semi-leptonic $t\bar{t}$ events). We have thus investigated the performance of the Isolation template method in the presence of W events. In this case, one needs to add to the fit function a third component (template) to describe the Isolation shape for prompt electrons. We note here that an Isolation template for prompt electrons can be easily extracted using data-driven ways (e.g. the random cones technique) or even using a MC shape directly. 

The selection of electrons with offline $P_{T} > 10$ GeV implies that one needs to incorporate the low $\hat{p}_{T}$-bins of QCD MC samples. We therefore include both the QCD inclusive jets samples QCD\_pt30 and QCD\_pt80, with a weight normalized to $0.1 \text{pb}^{-1}$ of integrated luminosity\footnote{The luminosity chosen to normalize the QCD MC samples corresponds roughly to the available statistics of the QCD\_pt30 sample; so that QCD\_pt30 events are worked with a weight of $\approx 1$}. The W pythia sample is also normalized accordingly and included in the Selection. We finally repeat the Isolation template method, using the combined relative Isolation distribution, as previously described, in order to extract the number of fake (background) electron events that fall into the Signal Selection region (RelIso $<0.1$). Figure~\ref{fig:w_combIso_fit} (left) shows the combined fit for $P_{T}(e)>10 $ GeV in the presence of a W signal. The same fit is repeated on the right plot, where now a $\text{pfMET} < 20 $ GeV cut has been applied to the Selection, to suppress the $W \rightarrow e\nu$ component. The latter case descreases the error on the measurement due to smaller correlations in the fit parameters. 

Armed with this result, we next investigate the performance of the template method on fake electron events (in signal region RelIso $< 0.1$) in the presence of W events, and with increasing a cut on the hadronic activity in the event -namely applying successively an $H_{T}$ cut. Figure~\ref{fig:w_fitprediction_mc} shows a comparison of the number of background events observed in the signal region, in black, versus the number of fake electron events predicted in blue. Also superimposed are the number of total events, - including Ws -, in the signal region shown in red dots. A cut of $\text{pfMET} < 20$~GeV has been applied. Similar plots are repeated for electrons with $P_{T}> 20$ GeV (see Figures~\ref{fig:w_combIso_fit_pt20} and~\ref{fig:w_fitprediction_mc_pt20}).

%\clearpage

\begin{figure}[h!]
\centering
\includegraphics[scale=0.32]{Plots/w_combIso_pt10_fit_MC.png}
\includegraphics[scale=0.32]{Plots/w_combIso_pt10_METanticut_fit_MC.png}
\caption{\textit{The combined Isolation distribution from Monte Carlo simulation of background electrons - using the PYTHIA QCD\_pt30 and QCD\_pt80 samples -, and prompt electrons - using the PYTHIA $W \rightarrow e\nu$ sample, for $p_{T}(e)>$ 10 GeV. On the right plot, a pfMET $ < 20$~GeV anti-cut has been applied. The dashed lines are the two background components as extracted from the two Control Samples, whereas the total ``predicted'' background is the solid line.}}
\label{fig:w_combIso_fit}
\end{figure}

\begin{figure}[h!]
\centering
\includegraphics[width=80mm]{Plots/w_fitprediction_pt10_METanticut_vsHT_MC.png}
%\end{center}
\caption{\textit{The number of truth background electrons (in black dots) in signal region, RelIso $< 0.1$, are compared to the fit prediction (in blue stars), as a function of the cut in the hadronic activity of the event ($H_{T}$ cut). A cut of pfMET $<20$~GeV has been applied to suppress sources of prompt electrons (Ws here). The number of total truth electron events - including residual Ws - is also shown superimposed in red triangles. }}
\label{fig:w_fitprediction_mc}
\end{figure}

\begin{figure}[h!]
\centering
\includegraphics[scale=0.32]{Plots/w_combIso_pt20_fit_MC.png}
\includegraphics[scale=0.32]{Plots/w_combIso_pt20_METanticut_fit_MC.png}
\caption{\textit{Same as Figure~\ref{fig:w_combIso_fit} only this time for electrons with $P_{T}$ threshold at 20~GeV.  }}\label{fig:w_combIso_fit_pt20}
\end{figure}

\begin{figure}[h!]
\centering
\includegraphics[width=80mm]{Plots/w_fitprediction_pt20_METanticut_vsHT_MC.png}
%\end{center}
%\vspace{1mm}
\caption{\textit{Same as Figure~\ref{fig:w_fitprediction_mc} only this time for electrons with $P_{T}>$ 20~GeV. }}
\label{fig:w_fitprediction_mc_pt20}
\end{figure}


Given the remarkably good description of the combined background in Monte Carlo simulation, we next test this procedure on CMS data.  This is the subject of the next section.

\subsection{Predicting the distribution of the $\alpha{_T}$ kinematic variable by inverting Electron ID Cuts}

Following the promising results of the $\alpha_{T}$ jet-balancing method previously described for the all-hadronic SUSY searches~\cite{njet}, a natural extension of this approach has been developed to the single-lepton SUSY search~\cite{ouratnote}, where a significant presence of QCD multi-jet backgrounds is expected.

The $\alpha_{T}$ variable is here defined as an N-object system where the set of objects is 1 electron and N-1 jets. This definition reproduces the kinematics of a di-jet system by contructing two pseudo-jets, which balance one another in $H_{T}$. The two pseudo-jets are formed from the combination of the N objects that minimizes the $\Delta H_{T} \equiv |H_{T,1} - H_{T,2}|$ of the pseudo-jets, and the resulting  $\alpha_{T}$ is
\begin{equation}
\alpha_{T} = \frac{1}{2} \frac{H_{T} - \Delta H_{T}}{M_{T}} =  \frac{1}{2} \frac{H_{T} - \Delta H_{T}}{\sqrt{H_{T}^{2}-MH_{T}^{2}}}.
\end{equation}

This section is dedicated to a first approach of commissioning the alphaT observable and study its behavior in pure fake electron events. It is therefore desirable to collect a suitable control sample which will be dominated by fake electrons and eliminate sources of prompt electron events (like W events).

One way to obtain such a sample is using the anti-selection method on electron ID variables which are less correlated with the missing transverse energy. In this section, we investigate the possibility of inverting the $\Delta \eta$(trk-SC) and $\Delta \phi$ (trk-SC) id cuts in the electron selection. The selected events in this method pass the pre-selection described in Section~\ref{sec:evtsel}, while the anti-selected are those events which pass the selection with an electron that passes all selection id criteria except the $\Delta \phi$ (trk-SC) and $\Delta \eta$(trk-SC) ones. 

In order to establish the validity of the control sample obtained by the anti-selection method above, the perfomance must be compared of the leptonic $\alpha_T$ as obtained from the control sample and the actual QCD events passing the electron criteria defined in the ``signal'' region. Because SUSY events are expected to have high $H_{T}$, it is desirable to understand how the method evolves with increasing $H_{T}$ cuts. The number of events for 1pb$^{-1}$ passing the pre-selection, and two different cuts in $H_{T}$ are shown for selected in Table \ref{tab:CF_S_20} and for anti-selected in Table \ref{tab:CF_AS_20}. 

\vspace{3mm}
\begin{table}[h!]
\begin{center}
\begin{tabular}{|c||c|c|c|c|c|}
\hline
Cutflow & QCD EM enriched & QCD BC$\rightarrow e$ & QCDJets Pythia $\hat{p_{T}}$ & W & \% contamination from W\\
\hline
\hline
All events & 5351938 & 256514.4 & 25470 & 24170 & 0.43\%\\
N($e^{-}$) $\geq$ 1 & 9372.1 & 2635.5 & 19.9 & 3848.6 & 32\%\\
N($jets$) $\geq$ 1 & 8174.5 & 2317.3 & 10.4 & 3632.48 & 34.6\%\\
HT $>$ 100 GeV & 249.2 & 55.8 & 6.2 & 95.3 & 30.6\%\\
HT $>$ 180 GeV & 7.53 & 1.2 & 3.6 & 2.86 & 23.2\%\\
\hline
\end{tabular}
\end{center}
\caption{\textit{Cutflow for selected events.Numbers shown for 1pb$^{-1}$ for both QCD and W samples used for the anaylsis with electron $p_{T}$ requirement set to 20GeV.}}
\label{tab:CF_S_20}
\end{table}

\begin{table}[h!]
\begin{center}
\begin{tabular}{|c||c|c|c|c|c|}
\hline
Cutflow & QCD EM enriched & QCD BC$\rightarrow e$ & QCDJets Pythia $\hat{p_{T}}$ &  W & \% contamination from W\\
\hline
All events & 5351938 & 256514.4 & 25470 & 24170 & 0.43\%\\
N($e^{-}$) $\geq$ 1 & 32725.5 & 729.0 & 52.9 & 298.0 & 0.89\%\\
N($jets$) $\geq$ 1 & 27545.3 & 609.9 & 28.1 & 271.5 & 0.96\%\\
HT $>$ 100 GeV & 902.8 & 24.9 & 18.3 & 4.9 & 0.52\%\\
HT $>$ 180 GeV & 20.7 & 0.8 & 10.7 & 0.3 & 1.01\%\\
\hline
\end{tabular}
\end{center}
\caption{\textit{Cutflow for anti-selected events.Numbers shown for 1pb$^{-1}$ for both QCD and W samples used for the anaylsis with electron $p_{T}$ requirement set to 20GeV.}}
\label{tab:CF_AS_20}
\end{table}

It can be seen that the anti-selected control sample allows one to study and validate the expected behavior of the $\alpha_{T}$ in a sample of \textit{pure} fake electron events, where the contamination from prompt electrons should be at a level below 1\%, as opposed to the selection region where $W \rightarrow e\nu$ is expected to contribute at the $\approx 10$\% level.

\subsubsection{Closure test with pure QCD background sample}

The control sample provided by anti-selection on the ($\Delta \eta$/$\Delta \phi$) electron id variables must first undergo a closure test with a pure QCD sample. This will show if there is any bias in QCD between the distribution of $\alpha_{T}$ in the selected and the anti-selected regions. The plots in \ref{fig:AlphaTbyHT} show the normalised shape of distributions, firstly before an $H_{T}$ cut (top left) and then as increasing cuts in $H_{T}$ are applied. The selected and anti-selected distributions show good agreement. The evolution of the $\alpha_{T}$ as $H_{T}$ cut increases shows the expected reduction in the tail for $\alpha_{T} >$ 0.55, the region of likely SUSY signal, for both selected and anti-selected events.

\begin{figure}[h!]
\includegraphics[width=50mm]{Plots/mc-alphaT-1}
\includegraphics[width=50mm]{Plots/mc-alphaT-2}
\includegraphics[width=50mm]{Plots/mc-alphaT-3}
\includegraphics[width=50mm]{Plots/mc-alphaT-4}
\hspace*{3mm}
\includegraphics[width=50mm]{Plots/mc-alphaT-5}
\hspace*{3mm}
\includegraphics[width=50mm]{Plots/mc-alphaT-6}

\caption{\textit{The $\alpha_{T}$ distributions for selected (red) and anti-selected events (black) for the QCD multi-jet background, from inversion of the $\Delta \phi$ and $\Delta \eta$ ID Cuts, shown without $H_{T}$ cut (Top Left) and with progressive HT cuts (left-right, top-bottom). These distributions are normalised to unity for shape comparison. There is good agreement between the selected and anti-selected samples regardless of HT requirement, and the high $\alpha_{T}$ tails reduce as expected when moving to higher HT cuts.}}
\label{fig:AlphaTbyHT}
\end{figure}

In order to demonstrate the power of $H_{T}$ in $\alpha_{T}$ tail-reduction, we introduce the variable $R_{\alpha_T}$ which is defined as the ratio of the number of events passing the $\alpha_T$ cut over the number of events failing it:
\begin{equation}
R_{\alpha T} = \frac{N(\alpha_{T}>0.55)}{N(\alpha_{T}>0.)}
\end{equation}
The ``default'' cut value here is the value prompted from the all-hadronic analysis, 0.55. Figure \ref{fig:AlphaT_Ratio} shows a plot of $R_{\alpha_T}$ as a function of the $H_{T}$ cut applied. As the $H_{T}$ cut value increases, $R_{\alpha_T}$ is observed to decrease in an (approximately) exponential manner~\cite{david}. This result confirms that the noticable reduction in the tail are much more pronounced than those in the peak, therefore this is not a statistical effect only. The selected and anti-selected events remain in good agreement.

\begin{figure}[h!]
\begin{center}
\includegraphics[width=80mm]{Plots/mc-alphaTratio}
\end{center}
\caption{\textit{The $R_{\alpha_T}$ versus the $H_{T}$ cut applied for the QCD multi-jet background, shown for both selected and anti-selected events in the Delta ID Inversion method.}}
\label{fig:AlphaT_Ratio}
\end{figure}

\subsubsection {Closure test with W + jets contamination in control region}

Having acertained the validity of the anti-selection method to predict the QCD contribution in the selected from the QCD in the anti-selected, it is important to test whether the process will work with contamination in the control region. The method is designed for data-driven extimation, and thus must be robust to such contamination.

The closure test is repeated, with the anti-selected now from both the pure QCD sample as before, and W + jets also. The selected remains from pure QCD as a comparison. The normalised ditribution plots shown for this case are in Figure~\ref{fig:w-AlphaTbyHT}, and the plot of $R_{\alpha_T}$ as a function of the $H_{T}$ cut applied is in Figure~\ref{fig:w-AlphaT_Ratio}. 

\begin{figure}[h!]

\includegraphics[width=50mm]{Plots/w-alphaT-1}
\includegraphics[width=50mm]{Plots/w-alphaT-2}
\includegraphics[width=50mm]{Plots/w-alphaT-3}
\includegraphics[width=50mm]{Plots/w-alphaT-4}
\hspace*{3mm}
\includegraphics[width=50mm]{Plots/w-alphaT-5}
\hspace*{3mm}
\includegraphics[width=50mm]{Plots/w-alphaT-6}
\caption{\textit{Same as in Figure~\ref{fig:AlphaTbyHT} only this time with W contamination in the control region.}}
\label{fig:w-AlphaTbyHT}
\end{figure}

%Figure \ref{fig:AlphaT_Ratio} shows a plot of $R_{\alpha_T}$ as a function of the $H_{T}$ cut applied. As the $H_{T}$ requirement increases, $R_{\alpha_T}$ decreases in an exponential manner. Thic confirms that the noticable reduction in the tail are much more pronounced than those in the peak, therefore this is not a statistical effect only. The selected and anti-selected events remain in good agreement.

\begin{figure}[h!]
\begin{center}
\includegraphics[width=80mm]{Plots/w-alphaTratio}
\end{center}
\caption{\textit{Same as in Figure~\ref{fig:AlphaT_Ratio} only this time with W contamination in the control region.}}
%The $R_{\alpha_T}$ versus the $H_{T}$ cut applied for the QCD multi-jet background, with contamination from W + jets in the anti-selected only, shown for both selected and anti-selected events in the Delta ID Inversion method.}}
\label{fig:w-AlphaT_Ratio}
\end{figure}

Adding in contamination from the W + jets sample has no discernable effect on the shape of the $\alpha_{T}$ distribution, and the ratio remains to good agreement also. The control sample is thus robust to such a contamination, and still remains a good estimator of the shape of the QCD background for the selected events.

\begin{comment}
\subsubsection{Closure test with lowered electron $P_{T}$ threshold}
It is also important to commission $\alpha_{T}$ as the electron $p_{T}$ requirement is lowered. The study is repeated with the electron $p_{T} >$ 10 GeV. Tables \ref{tab:CF_S_10} and \ref{tab:CF_AS_10} show the numbers for selected and anti-selected cutflows respectively at $1pb^{-1}$. 

\begin{table}[h!]
\begin{center}
\begin{tabular}{|c||c|c|c|c|}
\hline
Cutflow & QCD EM enriched & QCD BC$\rightarrow e$ & QCDJets Pythia $\hat{p_{T}}$ &  W \\
\hline
All events & 5351938 & 256514.4 & 25470 & 17830 \\

N($e^{-}$) $\geq$ 1 & 22120.98 & 13117.37 & 4.37 & 2953.69\\

N($jets$) $\geq$ 1 & 14783.29 & 7811.25 & 4.37 & 761.22\\

HT $>$ 100 GeV & 846.64 & 312.95 & 4.37 & 78.73\\

HT $>$ 180 GeV & 159.83& 36.05 & 4.29 & 18.20\\
\hline
\end{tabular}
\end{center}
\caption{\textit{Cutflow for selected events.Numbers shown for 1pb$^{-1}$ for both QCD and W samples used for the anaylsis with electron $p_{T}$ requirement set to 10GeV.}}
\label{tab:CF_S_10}
\end{table}

\begin{table}[h!]
\begin{center}
\begin{tabular}{|c||c|c|c|c|}
\hline
Cutflow & QCD EM enriched & QCD BC$\rightarrow e$ & QCDJets Pythia $\hat{p_{T}}$ &  W \\
\hline
All events & 5351938 & 256514.4 & 25470 & 17830\\
N($e^{-}$) $\geq$ 1 & 44807.39 & 1406.28 & 3.30 & 173.04\\
N($jets$) $\geq$ 1 & 30089.89 & 920.40 & 3.30 & 43.52\\
HT $>$ 100 GeV & 1291.41 & 51.64 & 3.29 & 4.05\\
HT $>$ 180 GeV & 112.68 & 5.55 & 3.20 &  0.92\\
\hline
\end{tabular}
\end{center}
\caption{\textit{Cutflow for anti-selected events.Numbers shown for 1pb$^{-1}$ for both QCD and W samples used for the anaylsis with electron $p_{T}$ requirement set to 10GeV.}}
\label{tab:CF_AS_10}
\end{table}

\end{comment}
