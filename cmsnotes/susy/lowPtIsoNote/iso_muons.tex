\section{The strange case of isolated muons}
\label{app:isomuons}

It has been observed that there is a sharp peak at zero in the isolation
distributions both for fake muons and those coming heavy flavour decays. Since
this is a surprising result and could reduce the effectiveness of an isolation
cut in supressing the QCD background, it was important to understand the causes
of this effect. In order to do this, a generator level study was undertaken.

For a reconstructed muon, we may calculate a generator level tracker isolation
quantity GenIso. This is defined as follows:
\begin{equation}
\textrm{GenIso}=\sum_{0.01<\Delta R <0.3}p_{\textrm{T}}^{\textrm{\tiny{GEN}}}
\end{equation}
where the sum is taken over generator level particles with $p_T > 200$~ MeV
(this matches the threshold in offline track reconstruction). Figure~\ref{???}
shows a comparison of GenIso and absolute tracker isolation for muons coming
from heavy flavour decays. It can be seen that GenIso shows very similar
behaviour to the tracker isolation and thus the zero peak is a genuine effect. 

By investigating the origins of the generator level muon and the particles
contributing to the isolation, it was found that two effects contribute towards
the peak. Firstly, in a small number of cases, jet fragmentation will lead to
zero charged particles emerging from the decay of the b quark. In this case,
even if the muon is colinear with the jet, the tracker isolation will be zero
(this does not explain the fact that zero peaks are also seen in the ECAL and
HCAL isolations). The second effect is from asymmetric b decays where either the
muon or associated D meson takes a larger share of the $p_T$ of the B
meson. This can cause the muon to be separated from the jet and also lead to
zero isolation. In many cases, these effects will combine accounting for the
$\sim 3$\% of events seen in the zero isolation peak.
