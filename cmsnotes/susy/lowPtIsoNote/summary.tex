\section{Summary}
\label{sec:Summary}
%In this note we presented the first isolation study in CMS 
%devoted to soft leptons (below 30 GeV) to be used in leptonic Susy searches.
%We decided to apply only the track-based isolation for two reasons:
%its performances are better than calorimetric and combined isolation;
%low-$p_T$ tracks are more reliable than low-energetic calorimetric deposits
%especially in the start-up scenario.
%We proposed four different optimization strategies, and for each of them 
%a set of cut values and expected performances are provided.
%We applied the proposed selection in the Susy search with single 
%lepton and same-sign dilepton final states.
%In both the cases the improvement with respect the V+jets prescription
%in the significance(15-30\%) is remarkable.
%We also showed that with the proposed isolation we will be able to lower significantly 
%the lepton momentum threshold(from 10 to 5 GeV) without including too much background. \\
%In order to further improve HF leptons rejection, 
%an ecal-based isolation can be applied on top of tracker isolation.
%The expected efficiency and rejection power after the ECAL isolation are given in the text.

First CMS study of lepton isolation in $p_T$ region below 30 GeV is 
presented
to be used in SUSY analysis. Isolation variables based on tracking, ECAL 
and
HCAL information separately as well as combined isolation were examined.
The track-based isolation procedure demonstarated more efficient 
performance than
other types of isolation, which makes it more attractive for the low-$p_T$ 
regime,
especially in the CMS start-up scenario. An additional rejection power
of the isolation procedure can be achived by applying ECAL isolation cuts.

Four different isolation cuts optimization strategies are investigated
and set of cut values and expected performances are provided.
The proposed selections were applied to Single Lepton and Samie Sign 
Dilepton SUSY analyses. In both cases the improvement with respect to the 
standard V+jets prescription
in the significance(15-30\%) is observed.
It is also demonstrated that with the proposed isolation
the lepton momentum can be lowered from 10 to 5 GeV without lost  in 
signal-to-background significance.
