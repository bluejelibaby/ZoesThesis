\section{Introduction \label{sec:intro}}

In this note we present an update of the search for a missing energy
signature in dijet and multijet events using the \alt variable.  The
current results are based on 1.1~\fbinv of LHC data recorded in 2011
at a centre-of-mass energy of $\sqrt{s} = 7 \tev$.

The presented search concentrates on event topologies in which heavy
new particles are pair-produced in a proton-proton collision and where
at the end of their decay chain a weakly interacting massive particle
(WIMP) is produced. The latter remains undetected, thus leading to a
missing energy signature. In the case of SUSY, squarks and gluinos
could be the heavy particles while the lightest (and stable)
neutralino $\chiznew_1$ is the WIMP candidate. Although this search is
carried out in the context of SUSY, the results are applicable to
other New Physics scenarios as the missing energy signature is common
to many models, e.g., Extra Dimensions and Little Higgs models.

To interpret the results, a simplified and practical model of
SUSY-breaking, the constrained minimal supersymmetric extension of the
standard model (CMSSM)~\cite{ref:CMSSM, ref:MSUGRA}, is used.  The
CMSSM is described by five parameters: the universal scalar and
gaugino mass parameters ($m_0$ and $m_{1/2}$, respectively), the
universal trilinear soft SUSY breaking parameter $A_0$, and two
low-energy parameters, the ratio of the two vacuum expectation values
of the two Higgs doublets, $\tan\beta$, and the sign of the Higgs
mixing parameter, $\sign(\mu)$.  Throughout this note, two CMSSM
parameter sets, referred to as LM4 and LM6~\cite{PTDRII} and not
excluded by the 2010 analysis~\cite{RA1Paper}, are used to illustrate
possible CMSSM yields.  The parameter values defining LM4 (LM6) are
$m_0=210\gev$, $m_{1/2}=285\gev$, ($m_0=85\gev$, $m_{1/2}=400\gev$)
and $A_0=0$, $\tan\beta=10$, and $\sign(\mu)>0$ for both points.

Events with $n$ high-\pt hadronic jets are studied and the missing
transverse momentum is inferred through the measured jet momenta via
the kinematic variable \alt, which was initially inspired by
Ref.~\cite{Randall:2008rw}.  The analysis follows closely
Ref.~\cite{RA1Paper} and two previous Physics Analysis
Summaries~\cite{cms-pas-sus-09001,cms-pas-sus-08005}.  The main
difference with respect to Ref.~\cite{RA1Paper} is that rather than
defining a specific signal region, we now search for an excess of
events in data over the Standard Model expectation over the entire
$\scalht = \sum_{i=1}^{n} \Et^{ {\rm jet}_i}$ range above $275 \gev$.
This approach is complementary to the searches carried out in
Refs.~\cite{RA2} and~\cite{Razor}.  The dominant background which
arises from QCD multi-jets can be suppressed significantly with a
selection requirement on the \alt variable.  To estimate the remaining
backgrounds we make use of data control samples. These are a $\mu$ +
jets sample for the background from \wj and \ttbar events, and a
photon+jets sample to determine the background from \znunu\ events.

This note is organized as follows: In Section~\ref{sec:datasamples}
the data and Monte Carlo (MC) samples are listed. The event selection
and analysis method is described in Section~\ref{sec:selection},
followed by a discussion of the data-driven methods that were
developed for background estimation. Systematic uncertainties on the
signal efficiency are discussed in Section~\ref{sec:systematics}. The
statistical method used for the interpretation of the results is
discussed in Section~\ref{sec:statistics} and limits in the CMSSM
parameter space are presented in Section~\ref{sec:results} before
concluding.



