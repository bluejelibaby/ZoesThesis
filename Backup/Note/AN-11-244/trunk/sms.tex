\subsection{Cross-section limits for Simplified Model Spectra \label{sec:sms}}

The following description of the Simplified Model Spectra is taken
from Refs.~\cite{SMSAN,SMSnote} where they are described in detail.

In short, they are~\footnote{So far only the two models listed below have been considered. 
Additionally, models with pair-produced gluinos or squarks decaying through a one stage cascade
resulting in jets, LSP and a W are available and could be investigated as well if desired.}:

\begin{itemize}
\item pair-produced gluinos where each gluino directly decays to two light quarks and the LSP;
\item pair-produced squarks where each squark decays to one jet and the LSP;
%\item pair-produced gluinos or squarks that decay via a wino and jets and subsequently decaying to W. 
\end{itemize}

\begin{figure}[tbp]
  \begin{center}
      \includegraphics[angle=90,width=0.3\textwidth]{figures/sms/T1.pdf}
      \includegraphics[angle=90,width=0.3\textwidth]{figures/sms/T2.pdf}
%  \end{center}
%  \begin{center}
%      \includegraphics[angle=90,width=0.3\textwidth]{figures/sms/T3.pdf}
%      \includegraphics[angle=90,width=0.3\textwidth]{figures/sms/T4.pdf}
      \caption{Diagram of simplified models. Top left: gluino pair
        production; top right: squark pair production.
%        ; bottom row: one
%        single step cascade decay for gluino pair production and
%        squark pair production.
}
    \label{fig:diagram}
  \end{center}
\end{figure}


Figure~\ref{fig:diagram} shows the respective diagrams for the simplified topologies.
Fast simulation Monte Carlo samples are generated for different combinations of squark 
(gluino) and LSP masses.

%For the one stage decay, we consider only events where the W decays hadronically 
%(excluding the $\tau$ final state). There are numerous advantages in doing this. 
%First, limits/fits can be readily translated to apply to processes with Z in the 
%vast majority of parameter space. Second the results can be easily combined with 
%the results from the leptonic searches as the two signals are are exclusive. 
%Third the impact of the systematics uncertainties coming from the lepton veto 
%efficiency estimation in the context of hadronic searches is in this way minimized.


% In the case of cascades, there is an additional mass that affects the final state 
% kinematics and the acceptance for the signal. In this case, Monte Carlo samples are 
% generated for different intermediate masses of $m_{\chi}$ for each $m_{LSP}$ and $m_{gluino}$.
% The $m_{gluino}$-$m_{\chi}$ difference is proportional to the jet \pt emitted directly from 
% the gluino and the $m_{\chi}$-$m_{LSP}$ mass difference is proportional to the energy of the 
% jets from the W.

% As suggested in~\cite{SMSnote}, the following slices are studied separately: 
% $m_{\chi}$=1/2($m_{LSP}$+$m_{gluino}$) and $m_{\chi}$=($m_{LSP}$+1/4 $m_{gluino}$) or 
% $m_{\chi}$=($m_{LSP}$+3/4 $m_{gluino}$) and $m_{\chi}$=$m_{LSP}$+82.
% \FIXME{Double check what we use!}



\begin{figure}[t]
   \begin{center}
     \subfigure[\label{fig:xsecT1}]{\includegraphics[angle=0, width = 0.45\textwidth]{figures/sms/profileLikelihood_T1_lo_2HtBins_constantR_xsLimit.pdf}}\
     \subfigure[\label{fig:xsecT1}]{\includegraphics[angle=0, width = 0.45\textwidth]{figures/sms/profileLikelihood_T1_lo_2HtBins_constantR_xsLimit_logZ.pdf}}\
     \subfigure[\label{fig:effT1}]{\includegraphics[angle=0, width = 0.45\textwidth]{figures/sms/T1_eff.pdf}}
   \caption{\label{fig:T1} 
 Top: Cross-section for the T1 topology excluded at the 95\% CL. Bottom: efficiency times acceptance of the analysis for the T1 topology.
}
\end{center}
\end{figure}

\begin{figure}[t]
   \begin{center}
     \subfigure[\label{fig:xsecT2}]{\includegraphics[angle=0, width = 0.45\textwidth]{figures/sms/profileLikelihood_T2_lo_2HtBins_constantR_xsLimit.pdf}}\
     \subfigure[\label{fig:xsecT2}]{\includegraphics[angle=0, width = 0.45\textwidth]{figures/sms/profileLikelihood_T2_lo_2HtBins_constantR_xsLimit_logZ.pdf}}\
     \subfigure[\label{fig:effT2}]{\includegraphics[angle=0, width = 0.45\textwidth]{figures/sms/T2_eff.pdf}}
   \caption{\label{fig:T2} 
 Top: Cross-section for the T2 topology excluded at the 95\% CL. Bottom: efficiency times acceptance of the analysis for the T2 topology.
}
\end{center}
\end{figure}

% \begin{figure}[t]
%    \begin{center}
% %   \subfigure[\label{fig:effT3}]{\includegraphics[angle=90, width = 0.45\textwidth]{figures/sms/T3_eff.pdf}}
% %     \subfigure[\label{fig:xsecT3}]{\includegraphics[angle=90, width = 0.45\textwidth]{figures/sms/T3_xsec.pdf}}\
%    \caption{\label{fig:T3} 
% Left: efficiency times acceptance of the analysis for the T3 topology. Right: Cross-section for the T3 topology excluded at the 95\% CL.
% }
% \end{center}
% \end{figure}

% \begin{figure}[t]
%    \begin{center}
% %   \subfigure[\label{fig:effT4}]{\includegraphics[angle=90, width = 0.45\textwidth]{figures/sms/T4_eff.pdf}}
% %     \subfigure[\label{fig:xsecT4}]{\includegraphics[angle=90, width = 0.45\textwidth]{figures/sms/T4_xsec.pdf}}\
%    \caption{\label{fig:T4} 
% Left: efficiency times acceptance of the analysis for the T4 topology. Right: Cross-section for the T4 topology excluded at the 95\% CL.
% }
% \end{center}
% \end{figure}



Figs.~\ref{fig:T1} and \ref{fig:T2} show the efficiency and upper
limit on the cross-section for the T1 and T2 topologies, respectively.
There is not a one-to-one relation between efficiency and cross
section limit because of signal contamination in the background
control samples.

It can be seen that the efficiency of the analysis is much reduced in
regions of parameter space where the squark (gluino) and LSP masses 
are similar as in this case the production of hard jets is suppressed. 
It is highest for heavy squarks (gluinos) and LSP mass roughly half
the squark (gluino) mass which leads to hard jets and sizeable missing
energy.


