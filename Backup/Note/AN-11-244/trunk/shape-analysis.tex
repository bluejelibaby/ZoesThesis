\subsection{Improved sensitivity to higher-mass states \label{sec:shape-analysis}}

The signal region in the RA1 analysis is defined by $\scalht > 350
\gev$.  For reasons of simplicity and robustness it was decided to
carry out a simple cut-and-count interpretation of the final results
for the RA1 publication. In order to gain additional sensitivity to
higher-mass states it is possible to make a re-interpretation of the
observed and expected number of events by splitting the original
signal region into multiple \scalht bins. Therefore, the simple
cut-and-count interpretation of one signal bin becomes a \scalht shape
analysis based on several bins. This requires that the data driven
background methods used to determine the expected number of SM
background events in the signal region provide an estimate for each of
the \scalht bins in the signal region of $\scalht > 350 \gev$. The
data driven background methods used for the RA1 analysis were
explicitly designed with this use-case in mind. In the following, the
results for the different background prediction methods are presented
when splitting the signal region into two bins: $350 < \scalht < 450
\gev$ and $\scalht > 450 \gev$, which is the optimal number of bins
that can be exploited for the available data set. It should be noted
that this approach does not involve any changes of cuts or even a
re-analysis of the data. It simply represents a re-interpretation of
the final results in terms of multiple \scalht bins instead of only
one.


\subsubsection{Total background estimation using \scalht dependence of \RaT  \label{sec:ht-scaling}}

The total Standard Model background is estimated based on an
extrapolation of the ratio $\RaT = N^{\alt > 0.55}/N^{\alt < 0.55}$
from a low \scalht control region to the signal region $\scalht > 350
\gev$.

As in Ref.~\cite{RA1Paper}, the following dependencies of \RaT on
\scalht are considered:
\begin{itemize}
\item a constant behaviour
\item an exponential dependence
\end{itemize}

For the RA1 publication the more conservative approach of assuming an
exponential dependence was chosen as the default method, while the
more precise constant assumption was used as a cross check.
Table~\ref{tab:altratio} shows how the predicted and observed number
of events are divided between the two \scalht bins in the signal
region. Figure~\ref{fig:altratio} displays the observed \scalht
evolution of \RaT for data, SM and SM+LM1. Also superimposed on each
plot are $\pm$1$\sigma$ bands representing the expected evolution of
\RaT as a function of \scalht, based on the measured \RaT values in
the control \scalht region and the assumption of exponential or
constant dependence on \scalht.

% \begin{table}[ht] 
% \caption{Observed and predicted event yields in the different \scalht regions. The quoted uncertainties are statistical only.}
% \label{tab:altratio} 
% \begin{center}
% \begin{tabular}{l||c c|c c| c}
% \hline
% \scalht (\gev)           & 250-300         & 300-350         & 350-400         & 400-450         & $>$450 \\\hline
% $N^{\alt>0.55}$          & 33              & 11              & 5               & 3               & 5      \\  
% $N^{\alt<0.55}$          & 844459          & 331948          & 146126          & 79523           & 65475  \\\hline
% \RaT ($10^{-5}$) (Data)  & 3.91 $\pm$ 0.68 & 3.31 $\pm$ 1.00 & 3.42 $\pm$ 1.53 & 3.77 $\pm$ 2.18 & 7.64 $\pm$ 3.42 \\\hline
% \RaT ($10^{-5}$) (const) &\multicolumn{2}{c|}{ 3.72 $\pm$ 0.56 } & \multicolumn{2}{c|}{ 3.72 $\pm$ 0.56 } & 3.72 $\pm$ 0.56 \\
% %$N^{\alt>0.55}$ (pred1)  &      --         &   --            & 5.44 $\pm$ 0.82 & 2.96 $\pm$ 0.45 & 4.09 $\pm$ 0.62 \\\hline
% $N^{\alt>0.55}$ (const)  &      --         &   --            & \multicolumn{2}{c|}{8.39 $\pm$ 1.27} & 4.09 $\pm$ 0.62 \\\hline
% \RaT ($10^{-5}$) (exp)   & 3.91 $\pm$ 0.68 & 3.31 $\pm$ 1.00 & 2.81 $\pm$ 1.76 & 2.38 $\pm$ 2.31 & 2.02 $\pm$ 2.66 \\
% %$N^{\alt>0.55}$ (pred2)  &      --          &  --            & 4.11 $\pm$ 2.58 & 1.89 $\pm$ 1.84 & 2.22 $\pm$ 2.92 \\\hline
% $N^{\alt>0.55}$ (exp)    &      --          &  --            & \multicolumn{2}{c|}{6.00 $\pm$ 4.41} & 2.22 $\pm$ 2.92 \\\hline
% \end{tabular}
% \end{center}
% \end{table}


\begin{table}[ht]
  \caption{Observed and predicted event yields in the different
    \scalht bins of both the control and signal regions. The quoted uncertainties are statistical only.}
  \label{tab:altratio}
  \begin{center}
    \begin{tabular}{l|c c|c c}
      \hline
      \scalht (\gev)                & 250-300                  & 300-350                  & 350-450                  & $>$450 \\\hline
      $N^{\alt>0.55}$                & 33                            & 11                            & 8                              & 5      \\
      $N^{\alt<0.55}$                & 844459                    & 331948                    & 225649                    & 110036  \\\hline
      \RaT ($10^{-5}$) (Data)  & 3.91$^{+0.72}_{-0.64}$ & 3.31$^{+1.11}_{-0.91}$ & 3.55$^{+1.42}_{-1.12}$ & 4.54$^{+2.39}_{-1.77}$ \\\hline
      \RaT ($10^{-5}$) (const) &\multicolumn{2}{c|}{ 3.72$^{+0.61}_{-0.52}$ }   & 3.72$^{+0.61}_{-0.52} \pm 9\%_{\rm syst}$ & 3.72$^{+0.61}_{-0.52} \pm 14\%_{\rm syst}$ \\
      $N^{\alt>0.55}$ (const)    &      --                       &   --                           & 8.40$^{+1.37}_{-1.18} \pm 9\%_{\rm syst}$ & 4.10$^{+0.67}_{-0.58} \pm 14\%_{\rm syst}$ \\\hline
      \RaT ($10^{-5}$) (exp)   & 3.91$^{+0.72}_{-0.64}$ & 3.31$^{+1.11}_{-0.91}$  & 2.66$^{+2.17}_{-1.79} \pm 17\%_{\rm syst}$ & 1.84$^{+3.06}_{-2.54} \pm 29\%_{\rm syst}$ \\
      $N^{\alt>0.55}$ (exp)      &      --                        &  --                            & 6.00$^{+4.89}_{-4.03} \pm 17\%_{\rm syst}$ & 2.02$^{+3.36}_{-2.79} \pm 29\%_{\rm syst}$ \\\hline
    \end{tabular}
  \end{center}
\end{table}



%\begin{table}[ht] 
%\caption{Observed and predicted event yields in the different \scalht regions. The quoted uncertainties are statistical only.}
%\label{tab:altratio} 
%\begin{center}
%\begin{tabular}{l||c c|c c}
%\hline
%\scalht (\gev)           & 250-300         & 300-350         & 350-450         & $>$450 \\\hline
%$N^{\alt>0.55}$          & 33              & 11              & 8               & 5      \\  
%$N^{\alt<0.55}$          & 844459          & 331948          & 225649          & 65475  \\\hline
%\RaT ($10^{-5}$) (Data)  & 3.91 $\pm$ 0.68 & 3.31 $\pm$ 1.00 & 3.55 $\pm$ 1.25 & 7.64 $\pm$ 3.42 \\\hline
%\RaT ($10^{-5}$) (const) &\multicolumn{2}{c|}{ 3.72 $\pm$ 0.56 } & 3.72 $\pm$ 0.56 & 3.72 $\pm$ 0.56 \\
%$N^{\alt>0.55}$ (const)  &      --         &   --            & 8.39 $\pm$ 1.27 & 4.09 $\pm$ 0.62 \\\hline
%\RaT ($10^{-5}$) (exp)   & 3.91 $\pm$ 0.68 & 3.31 $\pm$ 1.00 & 2.60 $\pm$ 2.04 & 2.02 $\pm$ 2.66 \\
%$N^{\alt>0.55}$ (exp)    &      --          &  --            & 5.87 $\pm$ 4.61 & 2.22 $\pm$ 2.92 \\\hline
%\end{tabular}
%\end{center}
%\end{table}


%\begin{figure}[t]
%   \begin{center}
%   \includegraphics[angle=90, width = 0.9\textwidth]{figures/altratio_bg.pdf}
%   \caption{\label{fig:altratio} 
%Top: \RaT as a function of \scalht for data (black), constant assumption of \RaT dependence on \scalht (red) and
%exponential assumption of \RaT dependence on \scalht (blue).
%Bottom: Number of observed and predicted events as a function of \scalht. Data are shown in black, constant prediction
%in red and exponential prediction in blue.
%}
%\end{center}
%\end{figure}

\begin{figure}[t]
  \begin{center}
    \includegraphics[width = 0.48\textwidth]{figures/Ratio_Multi2Incl_AlphaT055-data-6bins.pdf}
    \includegraphics[width = 0.48\textwidth]{figures/Ratio_Multi2Incl_AlphaT055-sm-6bins.pdf}
    \includegraphics[width = 0.48\textwidth]{figures/Ratio_Multi2Incl_AlphaT055-lm1-6bins.pdf}
    \caption{\label{fig:altratio} \RaT as a function of \scalht
      for data (left), SM (right) and SM+LM1 (bottom). The
      light and dark bands represent the expected \RaT values 
      ($\pm$1$\sigma$) for each of the \scalht bins in the signal
      region ($\scalht > 350~\GeV$), for the exponential and constant
      behaviours, respectively.}
  \end{center}
\end{figure}

\subsubsection{Estimation of background from \ttbar and W + jets events using a muon control sample \label{sec:mu}}

Table~\ref{tab:mu} shows the split of the muon control sample numbers
and corresponding background prediction in the two \scalht bins.

\begin{table}[ht] 
\caption{Observed number of events in data and MC simulation for the $\mu$ + jets control sample
and the MC expectation \ttbar/W + jets events in the hadronic signal sample.}
\label{tab:mu} 
\begin{center}
\begin{tabular}{l|c|c}
\hline
Sample                     & $350 < \scalht < 450 \gev$&  $\scalht > 450 \gev$ \\\hline
$\mu$ + jets (Data; $\mu$ sample)        & 5   & 2   \\
$\mu$ + jets (MC; $\mu$ sample)          & 4.1 & 1.9 \\ 
\ttbar/W + jets (MC; hadr. sample)       & 3.4 & 1.7 \\ \hline
$\tau (N^{\ttbar/W; had}_{MC}/N^{\ttbar/W; mu}_{MC})$ & 0.83 $\pm$ 30\% & 0.89 $\pm$ 30\% \\\hline
Predicted \ttbar/W + jets BG &  4.2 $\pm$ $^{+1.8}_{-2.1}$ $_{stat}$ $\pm$ 1.3 $_{syst}$ & 1.8 $\pm$ $^{+1.4}_{-1.8}$ $_{stat}$ $\pm$ 0.5 $_{syst}$ \\ \hline
\end{tabular}
\end{center}
\end{table}



\subsubsection{Estimation of background from \znunu + jets from photon + jets events\label{sec:photon}}

Table~\ref{tab:phot} shows the split of the photon control sample
numbers and corresponding background prediction in the two \scalht
bins.


\begin{table}[ht] 
\caption{Observed number of events in data and MC simulation for the photon + jets control sample
and the MC expectation for \znunu + jets events in the hadronic signal sample.}
\label{tab:phot} 
\begin{center}
\begin{tabular}{l|c|c}
\hline
Sample               & $350 < \scalht < 450 \gev$&  $\scalht > 450 \gev$ \\\hline
$\gamma$ + jets (Data.; $\mu$ sample) & 6   & 1   \\
$\gamma$ + jets (MC; $\mu$ sample)    & 4.4 & 2.1 \\ 
\znunu\ +jets (MC; hadr. sample)      & 2.6 & 1.5 \\ \hline
$\tau (N^{\znunu; had}_{MC}/N^{\gamma; \gamma}_{MC})$ & 0.59 $\pm$ 40\% & 0.71 $\pm$ 40\% \\\hline
Predicted \znunu\ BG      & 3.5 $\pm$ $^{+1.4}_{-1.6}$ $_{stat}$ $\pm$ 1.4$_{syst}$ & 0.7 $\pm$ $^{1.0}_{-0.7}$ $_{stat}$ $\pm$ 0.3$_{syst}$ \\\hline
\end{tabular}
\end{center}
\end{table}

\subsubsection{Impact of the \scalht shape interpretation on the 95\% CL exclusion limit in the CMSSM}

Figure~\ref{RA1_twobin_tanb3} shows a comparison of the 95\% CL
exclusion limit for the published cut-and-count (i.e. one signal bin)
interpretation of the data and the limit obtained from the two-bin
shape analysis interpretation. As expected, there is a significant
gain in the observed and expected exclusion limits, of approximately
20 GeV in $m_{1/2}$ for fixed $m_0$ when using the shape
interpretation of the final result.

\begin{figure}[h]
   \begin{center}
    \includegraphics[width=0.8\textwidth]{figures/cmssm/RA1_ExclusionLimit_tanb3_1HT2HTRexp.pdf} 
    \caption{\label{RA1_twobin_tanb3} Comparison of the
      observed %and expected
      $95\%$ CL exclusion contour in the CMSSM $m_0 - m_{1/2}$ plane
      ($\tan \beta = 3$, $A_0 = 0$, $\sign(\mu)>0$) for the default
      analysis and re-interpretation using two \scalht bins in the
      signal region. {\it Note: Since producing the expected limit with toy experiments is CPU time consuming, we present here the comparison for the observed limit which leads to the same conclusion. The plot will be replaced with a smoothed version showing the expected limit after the pre-approval.}}
\end{center}
\end{figure}


