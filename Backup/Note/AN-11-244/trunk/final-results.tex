\section{Final Results \label{sec:results}}

In this section we discuss the results of the statistical
interpretation of the \scalht dependent yield measurements in the
all-hadronic, W+jet, and Photon+Jet samples.  We consider the
following scenario: small QCD contamination mainly in the lower
\scalht bins described by a single exponential function (see Section
4.2.); and constant \RaT as a function of \scalht for the dominant EWK
background

The \scalht distributions shown in Figure~\ref{fig:hadronic},
~\ref{fig:muon}, and ~\ref{fig:photon} are the result of the
simultaneous fit to the \scalht dependent yield in the all-hadronic,
W+jet and photon+jet sample, respectively.  A good agreement between
the measured \scalht distributions and the best fit is observed in all
three distributions, indicating that the number of events found in
data is compatible with the SM background expectation predicted by the
fit.  With a p-value of 0.56, the hypothesis for the \scalht
dependences of \RaT is well reproduced (see Figure~\ref{fig:rat}).
However, both QCD fit parameters, $A_{QCD}=(1.4 \pm 1.9) \times
10^{-5}$ and $k_{QCD}=(5.2 \pm 5.6)\times 10^{-3}$, are compatible
with zero indicating that no significant QCD contamination has been
observed in the signal region.  The fit results are tabulated in
Table~\ref{tab:results-fit}.

We interpret this lack of signal as a limit on the allowed parameter
space of the CMSSM. In each point of the parameter scape, the SUSY
particle spectrum is calculated using SoftSUSY \cite{Allanach:2001kg},
and the signal events are generated at leading order (LO) with
\PYTHIA6.4~\cite{pythia}.  NLO cross sections, obtained with the
program Prospino~\cite{Beenakker:1996ch}, are used to calculate the
observed and expected exclusion contours. The simulated signal events
are re-weighted to match the average number of pile-up events observed
in data.

\begin{table}[ht!]
\caption{Fit Results for 1.1fb$^{-1}$. Since the QCD fit parameters are compatible with zero (see text), the listed QCD contributions in this table are also compatible with zero. }
\label{tab:results-fit}
\centering
\footnotesize
\begin{tabular}{ c|c|c|c|c }

\hline
\scalht Bin (GeV)       & 275--325                       & 325--375                       & 375--475                       & 475--575                       \\ [0.5ex]
\hline
W + $\ttNew$ background & 363.7                          & 152.2                          &  88.9                          &  28.8                          \\ 
$\znunu$ background     & 251.4                          & 103.1                          &  86.4                          &  26.6                          \\ 
QCD background          & 172.4                          &  55.1                          &  26.9                          &   5.0                          \\ \hline
Total Background        & 787.4                          & 310.4                          & 202.1                          &  60.4                          \\ 
Data                    & 782                            & 321                            & 196                            & 62                             \\ 

\hline
\scalht Bin (GeV)       & 575--675                       & 675--775                       & 775--875                       & 875--$\infty$                  \\ [0.5ex]
\hline
W + $\ttNew$ background &  10.6                          &   3.1                          &   0.6                          &   0.6                          \\ 
$\znunu$ background     &   8.7                          &   4.3                          &   2.5                          &   2.2                          \\ 
QCD background          &   1.0                          &   0.2                          &   0.1                          &   0.0                          \\ \hline
Total Background        &  20.3                          &   7.7                          &   3.2                          &   2.9                          \\ 
Data                    & 21                             & 6                              & 3                              & 1                              \\ 

\hline
\end{tabular}
\end{table}

Figure~\ref{fig:cmssm} shows the observed and expected 95\% CL
exclusion limits in the $m_0$-$m_{1/2}$ plane for $\tan \beta = 10$
and $A_0 = 0 \gev$ calculated with a two-sided profile likelihood
method.  Squark and gluino masses of 1.25 \TeV can be excluded for
values of the common scalar mass at the GUT scale $m_0 < 530
\gev$. When calculating the observed limit using the \cls
method~\cite{cls-pdg}, squark and gluino masses of 1.1 \TeV can be
excluded for $m_0 < 500 \gev$.

 \begin{figure}[t]
   \begin{center}
     \includegraphics[width = 0.48\textwidth]{figures/stats_plots/RQcdZero/hadronic_signal_fit_logy.pdf}
     \includegraphics[width = 0.48\textwidth]{figures/stats_plots/RQcdFallingExp/hadronic_signal_fit_logy.pdf}
     \caption{\label{fig:hadronic} \scalht distribution for events in the hadronic signal sample for scenario a) (left) and scenario b) (right). Shown are the events observed in data (black points), the outcome of the fit (blue line) and a breakdown of the individual background contributions as predicted by the control samples. A possible signal contribution from benchmark point LM6 is indicated as well (yellow line).}
   \end{center}
 \end{figure}

 \begin{figure}[t]
   \begin{center}
     \includegraphics[width = 0.48\textwidth]{figures/stats_plots/RQcdZero/muon_control_fit_logy.pdf}
     \includegraphics[width = 0.48\textwidth]{figures/stats_plots/RQcdFallingExp/muon_control_fit_logy.pdf}
     \caption{\label{fig:muon}  \scalht distribution for events selected in the muon control sample for scenario a) (left) and scenario b) (right).
  Shown are the events observed in data (black points), the outcome of the fit (blue line) and the MC expectation (dashed line). A possible signal
 contribution from benchmark point LM6 is indicated as well (yellow line).}
   \end{center}
 \end{figure}

 \begin{figure}[t]
   \begin{center}
     \includegraphics[width = 0.48\textwidth]{figures/stats_plots/RQcdZero/photon_control_fit_logy.pdf}
     \includegraphics[width = 0.48\textwidth]{figures/stats_plots/RQcdFallingExp/photon_control_fit_logy.pdf}
     \caption{\label{fig:photon} \scalht distribution for events selected in the photon control sample for scenario a) (left) and scenario b) (right). Shown are the events observed in data (black points), the outcome of the fit (blue line) and the MC expectation (dashed line).}
   \end{center}
 \end{figure}

 \begin{figure}[t]
   \begin{center}
     \includegraphics[width = 0.48\textwidth]{figures/stats_plots/RQcdZero/hadronic_signal_alphaT_ratio.pdf}
     \includegraphics[width = 0.48\textwidth]{figures/stats_plots/RQcdFallingExp/hadronic_signal_alphaT_ratio.pdf}
     \caption{\label{fig:rat} \RaT as a function of \scalht as observed in data (black points) and the results of the fit assuming different scenarios: a) (left) and b) (right) .}
   \end{center}
 \end{figure}

\begin{figure}[t]
  \begin{center}
    \subfigure[\label{fig:hadronic} \scalht distribution for events in the hadronic signal sample. Shown are the events observed in data (black points), the outcome of the fit (light blue line) and a breakdown of the individual background contributions as predicted by the control samples. A possible signal contribution from benchmark point LM6 is indicated as well (magenta line).]{\includegraphics[width = 0.48\textwidth]{figures/stats_plots/RQcdFallingExp/hadronic_signal_fit_logy.pdf}}
\hspace{0.3cm}
    \subfigure[\label{fig:muon}  \scalht distribution for events selected in the muon control sample. Shown are the events observed in data (black points) and the outcome of the fit (light blue line). A possible signal contribution from benchmark point LM6 is indicated as well (magenta
line).]{\includegraphics[width = 0.48\textwidth]{figures/stats_plots/RQcdFallingExp/muon_control_fit_logy.pdf}}\\
    \subfigure[\label{fig:photon} \scalht distribution for events
    selected in the photon control sample. Shown are the events observed in data (black points), the outcome of the fit (light blue line).]{\includegraphics[width = 0.48\textwidth]{figures/stats_plots/RQcdFallingExp/photon_control_fit_logy.pdf}}
\hspace{0.3cm} 
   \subfigure[\label{fig:rat} \RaT as a function of \scalht as observed in data (black points) and the result of the fit.]{\includegraphics[width = 0.48\textwidth]{figures/stats_plots/RQcdFallingExp/hadronic_signal_alphaT_ratio.pdf}}
    \caption{\label{fig:fitresult} Result of the combined fit to the
      hadronic, muon and photon samples.
}
  \end{center}
\end{figure}

\begin{figure}[t]
  \begin{center}
    \includegraphics[width = 0.90\textwidth]{figures/RA1_ExclusionLimit_tanb10.pdf}
    \caption{\label{fig:cmssm} Observed and expected 95\% CL exclusion
      contours in the CMSSM ($m_0, m_{1/2}$) plane ($\tan \beta = 10,
      A_0 = 0, \mu > 0$) using NLO signal cross sections using the
      Profile Likelihood (PL) method. The expected limit is shown with
      its 68\% CL range.  The observed limit using the $\cls$ method is
      shown as well.  }
  \end{center}
\end{figure}



