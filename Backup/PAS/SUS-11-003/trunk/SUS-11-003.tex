% Customizable fields and text areas start with % >> below.
% Lines starting with the comment character (%) are normally removed before release outside the collaboration, but not those comments ending lines

% svn info. These are modified by svn at checkout time.
% The last version of these macros found before the maketitle will be the one on the front page,
% so only the main file is tracked.
% Do not edit by hand!
\RCS$Revision: 71659 $
\RCS$HeadURL: svn+ssh://svn.cern.ch/reps/tdr2/notes/SUS-11-003/trunk/SUS-11-003.tex $
\RCS$Id: SUS-11-003.tex 71659 2011-07-25 11:14:15Z flaecher $
%%%%%%%%%%%%% ptdr definitions %%%%%%%%%%%%%%%%%%%%%
%\input{ptdr-definitions} %These have been replaced by the equivalent style file
\newcommand\rs{\raisebox{1.0ex}[-1.0ex]}
\newcommand{\ra}{\ensuremath{\rightarrow}}
\newcommand{\HT}{\ensuremath{H_{T}}}
\newcommand{\znunu}{\ensuremath{{\text Z} \ra \nu\bar{\nu}}}
\newcommand{\zmumu}{\ensuremath{{\text Z} \ra \mu\mu}}
\newcommand{\wmunu}{\ensuremath{{\text W} \ra \mu\nu}}
\newcommand{\wtaunu}{\ensuremath{{\text W} \ra \tau\nu}}
\newcommand{\dphi}{\ensuremath{\Delta \phi}}
\newcommand{\dphijj}{\ensuremath{\Delta \phi_{ j1,j2}}}
\newcommand{\Pt}{\ensuremath{{p_{\text T}}\xspace}}
\newcommand{\pts}{\ensuremath{p_{\text T}{\text s}}\xspace}
\newcommand{\Et}{\ensuremath{{E_{\text T}}\xspace}}
\newcommand{\ptjf}{\ensuremath{p_{\rm T}^{ {\rm j}_1} }}
\newcommand{\ptjs}{\ensuremath{p_{\rm T}^{ {\rm j}_2} }}
\newcommand{\ptjt}{\ensuremath{p_{\rm T}^{ {\rm j}_3} }}
\newcommand{\etajf}{\ensuremath{\eta^{ {\rm j}_1} }}
\newcommand{\etajs}{\ensuremath{\eta^{ {\rm j}_2} }}
\newcommand{\etajt}{\ensuremath{\eta^{ {\rm j}_3} }}
\newcommand{\ttj}{\ensuremath{\rm{t}\bar{\rm{t}} + jets}\xspace}
\newcommand{\wj}{\ensuremath{\rm W + jets}\xspace}
\newcommand{\zj}{\ensuremath{\rm Z + jets}\xspace}
\newcommand{\al}{\ensuremath{\alpha}}
\newcommand{\alt}{\ensuremath{\alpha_{\text{T}}}\xspace}
\newcommand{\etaabs}{\ensuremath{|\eta|}}
%\newcommand{\gev}{\ensuremath{\mathrm{\,Ge\kern -0.1em V}}}
\newcommand{\pb}{\ensuremath{pb^{-1}}}
\newcommand{\mjj}{\ensuremath{M_{\text{inv}}^{j1,j2}}}
%\newcommand{\ttbar}{\ensuremath{t\bar{t}}}
\newcommand{\chiznew}{\ensuremath{\chi^{0}}\xspace}
\newcommand{\chipnew}{\ensuremath{\chi^{+}}\xspace}
\newcommand{\sQuanew}{\ensuremath{\tilde{\rm q}}\xspace}
\newcommand{\sGlunew}{\ensuremath{\tilde{\rm g}}\xspace}
\newcommand{\ttNew}{\ensuremath{\rm{t}\bar{\rm{t}}}\xspace}
\newcommand{\tev}{\TeV}
%<TW date="30/10/2010">
%\newcommand{\Et}{E_{T}}
\newcommand{\combIso}{Iso_{\textrm{comb.}}}
\renewcommand{\arraystretch}{1.2}
\newcommand{\bigNum}[2]{#1 \, \times \, 10 \, ^{#2}}
%</TW>

\newcommand{\raT}{\ensuremath{R_{\alt}}}
\newcommand{\RaT}{\ensuremath{R_{\alt}}\xspace}
\newcommand{\GeV}{\textrm{GeV}}
\def\eslash{{\hbox{$E$\kern-0.6em\lower-.05ex\hbox{/}\kern0.10em}}}
\def\vecmet{\mbox{$\vec{\eslash}_T$}} %missing ET vector
\def\vecet{\mbox{$\vec{E}_\text{T}$}} % ET vector
\def\MET{\mbox{$\eslash_\text{T}$}\xspace}
\def\met{\mbox{$\eslash_\text{T}$}\xspace}
\def\mex{\mbox{$\eslash_\text{x}$}} %missing Ex
\def\mey{\mbox{$\eslash_\text{y}$}} %missing Ey
\def\mepar{\mbox{$\eslash_\parallel$}}
\def\meperp{\mbox{$\eslash_\perp$}}
\def\Zmm{Z \rightarrow \mu\mu}
\def\metvec{\mbox{$\vec{\met}$}\xspace}
\def\metvecrec{\mbox{$\vec{\met}^{\rm rec}$}\xspace}
\def\metvecgen{\mbox{$\vec{\met}^{\rm gen}$}\xspace}
\def\metgen{\mbox{$\met^{\rm gen}$}\xspace}
\def\metparl{\mbox{$\mepar^{\rm rec}$}\xspace}
\def\metperp{\mbox{$\meperp^{\rm rec}$}\xspace}
\def\deltamet{\mbox{$\Delta\met$}\xspace}
\def\pthat{\mbox{$\hat{p}_T$}\xspace}
\def\hslash{{\hbox{$H$\kern-0.8em\lower-.05ex\hbox{/}\kern0.10em}}}
\def\MHT{\mbox{$\hslash_\text{T}$}\xspace}
\def\mht{\mbox{$\hslash_\text{T}$}\xspace}
\def\sumet{\mbox{$\sum \rm{E}_\text{T}$}\xspace}
\def\scalht{\mbox{$H_\text{T}$}\xspace}
\def\etmiss{\mbox{$\eslash_\text{T}$}\xspace}
\def\htmiss{\mbox{$\hslash_\text{T}$}\xspace}
\def\mtt{\mbox{$\rm{M}_\text{T2}$}\xspace}
\def\rmec{\mbox{$R_{\mht/\met}$}\xspace}
\def\bdphi{\mbox{$\Delta\phi^{*}$}\xspace}
\def\bigeslash{{\hbox{$E$\kern-0.38em\lower-.05ex\hbox{/}\kern0.10em}}}
\def\bigmet{\mbox{$\bigeslash_T$}}
\def\bighslash{{\hbox{$H$\kern-0.6em\lower-.05ex\hbox{/}\kern0.10em}}}
\def\bigmht{\mbox{$\bighslash_T$}}
\def\cls{\mbox{CL$_s$}\xspace}
\def\incl{\includegraphics[width=0.49\linewidth]}
\def\inclrot{\includegraphics[angle=90,width=0.47\linewidth]}
\def\INCL{\includegraphics[angle=90,width=0.45\linewidth]}
\def\Incl{\includegraphics[angle=90,width=0.60\linewidth]}




%%%%%%%%%%%%%%%  Title page %%%%%%%%%%%%%%%%%%%%%%%%
\cmsNoteHeader{SUS-11-003} % This is over-written in the CMS environment: useful as preprint no. for export versions
% >> Title: please make sure that the non-TeX equivalent is in PDFTitle below
\title{Search for supersymmetry in all-hadronic events with $\alpha_{\rm T}$}

% >> Authors
%Author is always "The CMS Collaboration" for PAS and papers, so author, etc, below will be ignored in those cases
%For multiple affiliations, create an address entry for the combination
\address[neu]{Northeastern University}
\address[fnal]{Fermilab}
\address[cern]{CERN}
\author[cern]{The CMS Collaboration}

% >> Date
% The date is in yyyy/mm/dd format. Today has been
% redefined to match, but if the date needs to be fixed, please write it in this fashion.
% For papers and PAS, \today is taken as the date the head file (this one) was last modified according to svn: see the RCS Id string above.
% For the final version it is best to "touch" the head file to make sure it has the latest date.
\date{\today}
 
% >> Abstract
% Abstract processing:
% 1. **DO NOT use \include or \input** to include the abstract: our abstract extractor will not search through other files than this one.
% 2. **DO NOT use %**                  to comment out sections of the abstract: the extractor will still grab those lines (and they won't be comments any longer!).
% 3. **DO NOT use tex macros**         in the abstract: External TeX parsers used on the abstract don't understand them.

\abstract{
An update of the search for supersymmetry in events with
jets and missing transverse energy, based on the analysis in
Ref.~\cite{RA1Paper}, is presented. 
The results are based on a data sample of 1.1 fb$^{-1}$ of integrated luminosity 
recorded at $\sqrt{s}$ = 7 TeV. In this search, the variable $\alpha_{\rm T}$ is used as
the main discriminator between events with real and fake missing transverse energy,
and no excess of events over the Standard Model expectation is found. 
Given this agreement, exclusion limits in the parameter space of the
constrained minimal supersymmetric model are set. In this model, squark and gluino
masses of 1.1 TeV are
excluded for values of the common scalar mass at the GUT scale  
$m_0 < 0.5$ TeV. 
}

% Do not comment out the following hypersetup lines (metadata). They will disappear in NODRAFT mode and are needed by CDS.
% Also: make sure that the values of the metadata items are sensible. For APS submissions, they are automatically converted to APS keywords.
\hypersetup{%
pdfauthor={RA1 Team},%
pdftitle={Search for supersymmetry in all-hadronic events with alpha_T},%
pdfsubject={CMS},%
pdfkeywords={CMS, physics, jets, missing energy, SUSY, alpha_T}}

\maketitle %maketitle comes after all the front information has been supplied

%%%%%%%%%%%%%%%%%%%%%%%%%%%%%%%%  Begin text %%%%%%%%%%%%%%%%%%%%%%%%%%%%%
%% **DO NOT REMOVE THE BIBLIOGRAPHY** which is located before the appendix

\tableofcontents
\newpage

%\input Preface

\input intro

%\input framework

%\input datasamples

%\input selection

\input results

\input systematics

\input statistics

\input final-results

%\input Interpretation

%\input final-results

\input conclusions

%\input controlvars

%\input backgrounds

%\input systematics



%% **DO NOT REMOVE BIBLIOGRAPHY**
\bibliography{auto_generated}   % will be created by the tdr script.

%\clearpage
%\appendix
%\input appendix

