\chapter*{Abstract}
\noindent 
Diffraction techniques -- incorporating both x-rays and neutrons, single-crystals and powders -- have been developed to allow crystal structures to be determined
at high pressures with high accuracy.

For single-crystal neutron-diffraction studies, an data collection
strategy has been developed for use with a sapphire-anvil cell and
a position-sensitive detector and, combined with a newly developed
clamp-type pressure
cell, has allowed both the scope and accuracy of high-pressure structural
studies using neutron-diffraction techniques to be extended.
Application of the new techniques to high-pressure
structural studies of H-ordering systems of the
KH$_2$PO$_4$-type, strongly suggests that the H-atom site
separation, $\delta$, is a strong determinant of the
ordering temperature T$_c$ such that a hydrogenous
material and its deuterated analogue have the same T$_c$, within error,  at
the same $\delta$.
The results also suggest that the differences in T$_c$ between
different H-ordering systems are determined by the differences in $\delta$, and
that $\delta$ in all systems tends to $\sim$0.22\AA\ as T$_c$$\rightarrow$0K.

For single-crystal x-ray diffraction studies using a diamond-anvil
cell, analysis has shown that removal of intensity from either the incident or
diffracted beams by simultaneous diffraction of the diamonds can reduce the
intensity of sample reflections by up to 50\%. A data collection strategy to
detect this effect has been developed. The use of tungsten as a gasket material
has also been investigated, and provides a possible solution to the
long-standing problem of using AgK$_{\alpha}$ radiation for single-crystal studies.
Application of these newly developed techniques to a structural study of the
high-T$_c$
superconductor YBa$_2$Cu$_4$O$_8$, illustrates the improved accuracy now
available using single-crystal x-ray diffraction techniques and
AgK$_{\alpha}$  radiation. The results of this study
cast considerable doubts on the accuracy of previous
high-pressure structural studies
of high-T$_c$ superconductors.
\vspace{10mm}
\normalsize


