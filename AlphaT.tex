\chapter{SUSY Search Topology and the $\alpha_{T}$  variable}
\cite{PRL1fb}
\cite{DawsonSearch}
\cite{1fbnote}
\cite{aTHTdepstuart}

This thesis describes an inclusive search for new physics with a signature of significant missing energy and hadronic jets. Whilst 


\section{SUSY Search Toplogies}

As previously discussed in Chapter \ref{ch:theory}, SUSY models that conserve R-Parity and therefore indicate new physics at the TeV scale introduce a candidate particle for dark matter. As this LSP cannot be observed due to its weakly interacting nature, searching for it is analogous to a search for large missing energy in particle collisions. In the CMS detector reconstruction of all visible particles allows us to calculate the transverse component of this quantity, missing $E_{T}$ or \met. 

As there are many models to describe the exact nature of SUSY due to the unknown mechanism of SUSY breaking, it is desirable to design an experimental search which does not rely on any one in particular, or even on the assumption of SUSY. These are called ``inclusive" searches, and retain sensitivity to any new physics resulting in a new particle with the properties of a WIMP. The main feature is a requirement of a large quantity of \met along with final state objects (hadronic jets, leptons, photons). The search space is then divided into channels via the final state objects required, in order to perform orthogonal searches to increase sensitivity and to allow combination 

Discussion of SUSY on the whole and specific models such as mSUGRA are then used to quantify the reach of the search and to tune the cuts with Monte Carlo data. Where no new physics is found it can be useful to set limits on the parameters of such models, and in this thesis we will use mSUGRA for this purpose, along with test points in the mSUGRA phase space. However it is important to remember that the search itself remains open and sensitive to any WIMP candidate. 

Physics at the LHC will suffer from high background rates, especially those from QCD, and the main goal of any analysis is selecting the new physics events required whilst removing the background from Standard Model processes. Missing energy can be observed in events in two ways, real missing energy from the production of weakly interacting particles, such as neutrinos and LSP's, and fake missing energy which is a result of mismeasurment of objects, or missed objects. 

Having noted that the generic signal produced by any such new physics model is a large amount of \met, it might be assumed this forms the main variable to separate signal from background events. As \met is measured in the calorimeters, it can be affected by miscalibration and noise in the detector, thus is not robust for early physics at the LHC. 

To combat this issue there is also the quantity \mht which represents the vector sum of transverse momenta $p_{T}$ of the jets in the system, giving the hadronic missing energy analogous to \met in a hadronic search. However, there are limitations to the use of either of these quantities, as they are not robust to mismeasurments of the jets. 

\section{$\alpha_{T}$ in a di-jet system}

The first step in devising a SUSY search strategy begins with the simplest of channels, the ``diet" search with just two jets and missing energy corresponding to two missing neutralinos. Due to the low multiplicity it is easy to understand kinematically the situation in play. Instead of using \met as the discriminating variable, it is possible to obtain a higher signal to background ratio $\sfrac{S}{B}$ using a new variable proposed by Randall and Tucker-Smith, $\alpha$, defined in Equation \ref{eqn:alpha}~\cite{Randall}. 


\begin{equation}
\alpha = \frac{E_{T}^{j2}}{M_{inv}^{j1,j2}}
\label{eqn:alpha}
\end{equation}

The $E_{T}^{j2}$ is the transverse energy of the second jet (the lowest in energy) and $M_{inv}^{j1,j2}$ is the invariant mass of the dijet system. The design of this variable allows us to exploit the expected back-to-back nature of any diet from QCD. Thus a well-measured QCD event can only take values of $\alpha < 0.5$. In sharp contrast, a SUSY event can, due to the unseen neutralinos, produce jets in a similar direction with a low invariant mass giving rise to high values of $\alpha$.

The transverse variant of this variable, given in Equation \ref{eqn:alpha} makes use of the transverse mass $M_{T}$ of the two jets as opposed to the invariant mass.

\begin{equation}
\alpha-{T} = \frac{E_{T}^{j2}}{M_{T}} 
\label{eqn:alphat}
\end{equation}

In this case a well-measured QCD event will have exactly 0.5. While both show equally strong power of background discrimination, $\alpha_{T}$ has greater signal retention for certain mSUGRA points,\cite{PASaT} and therefore is deemed comparable or superior. It is upon this variable that the search strategy is formed. The presence of the second jet energy in the numerator also gives rise to one of the most important properties of this variable, its resilience to jet mismeasurment. If there is a large mismeasurment of one of the jets, the order could be inverted. As a perfectly measured QCD event yields \alt = 0.5, the cut chosen is \alt > 0.55 in order to take into account the finite resolution of the jet energy measurement.  

 It is worth noting also that in the massless limit this may be re-written in terms of the azimuthal angle between the two jets, $\Delta \phi$ as in Equation \ref{eqn:alphatphi}. This relationship indicates a high correlation, and thus a cut on \alt renders a further cut on $\Delta \phi$ negligible\cite{ANaT}.

\begin{equation}
\alpha_{T} = \frac{\sqrt{E_{T}^{j2}/E_{T}^{j1}}}{2(1- \textrm{cos} \Delta \phi)} 
\label{eqn:alphat}
\end{equation}


\section{$\alpha_{T}$ in a n-jet system}
More complicated decay processes result in hadronic signatures with more than two jets, generalised to the n-jet system, for example where a gluino-squark pair decay to produce three quarks and two LSP's. Following the success of the construction of the \a\t variable in the diet topology, the variable was extended to apply firstly to 3-jet systems and then in a general form applicable to an n-jet system, thus incorporating the full hadronic SUSY search channel\cite{ANnaT} . This is undertaken by modelling the system of $n$ jets as though it were a diet system, through the mathematical construction of two pseudo jets. Thus \alt can be calculated using the properties of the pseudo jets. 




\section{$\alpha_{T}$ search strategy}

