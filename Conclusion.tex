\chapter{Conclusion}
\label{ch:conclusion}

A comprehensive search for a final state with missing energy and jets motivated by $R$-Parity conserving supersymmetry is presented in this analysis. The analysis considers the first 1.1\,fb$^{-1}$ of 7~TeV data taken by the CMS detector at the LHC in 2011. Using an inclusive strategy which requires a final state with jets, no leptons or photons and significant missing energy targets new physics models in which a dark matter candidate is present. 

Due to the large background from QCD processes at the LHC there is a considerable background from fake missing energy due to mis-measurment. The use of a novel variable \alt is employed to effectively remove this component of the background. The additional backgrounds are estimated with the help of two dedicated control samples, of $\mu$ + jets and $\gamma$ + jets to estimate the $\tto$W and Z backgrounds respectively.  


 A shape analysis across eight bins of $\HT$ simultaneously in the signal region and two control regions is performed using a likelihood fit. The data agree very well with simulation and are found by the goodness-of-fit test to be consistent with the hypothesis of the Standard Model only. 

Having established that there is no distinction from the Standard Model hypothesis with this luminosity, the results are interpreted in the scope of the Constrained Minimal Supersymmetric Standard Model, in order to exclude regions of its parameter space. Using values of A$_{0}$ = 0, tan $\beta$ = 10 and sign$(\mu)$ = +, the m$_{0}$ - m${1/2}$ plane is probed using the CL$_{S}$ statistical method and an exclusion limit is set at a 95\% confidence level. 

The exclusion corresponds to a lower limit on equal gluino masses and the mean of the squark  masses at 1.1~TeV for the range m$_{0}$ $<$ 500 GeV, where the exclusion power is at its greatest. For higher values of m$_{0}$, where the gluino mass is much lower than that of the mean squark mass, the exclusion limit corresponds to a gluino mass of 0.5 TeV. 

At the time of publishing of these results, the exclusion limits far exceeded those set previously by collider experiments, expanding considerably the region of the CMSSM that is incompatible with experimental results.  

At the end of this thesis, in Chapter~\ref{ch:ra4} the effects of allowing more signal into the $\mu$ control sample is studied. At the present luminosity the limit remains unchanged by the removal of the transverse mass cut. The move to the leptonic definition of \alt also leaves the current limit unchanged, although with the inclusion of potential signal this would significantly increase the significance of signal events in the higher regions of \HT.  The recommendation for the next iteration of the analysis is to proceed with the dual-signal scenario using the leptonic \alt cut to increase the significance in this bin, while retaining the previous control definition for cross-checks. 

