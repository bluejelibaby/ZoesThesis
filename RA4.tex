\chapter{Extending the $\mu$ Control Sample to a Signal Sample}

In Chapter~\ref{ch:RA1}, the $\mu$ control sample was used effectively to predict the background contribution from W and \tto events. The $\mu$ likelihood's incorporation into the overall likelihood in order to interpret the hadronic results allowed for some small signal contamination. However it was in general viewed as a constraint on the ``signal" region of the hadronic selection. 

The cuts outlined in Section~\ref{sec:musel} are designed to select events from Standard Model W decays, hence minimising the contamination from signal. However, as the simultaneous fit includes the signal efficiency in the $\mu$ control sample it is possible to relax the cuts and allow more potential signal into the $\mu$ yield. The following work represents the author's personal investigation into the effect of increasing the chance for signal contamination on the eventual limit. 

\subsection{Relaxing the Cuts}



\subsection{Event Yields}
\subsection{Fit Results}
