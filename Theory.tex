\chapter{Theoretical Overview}

This analysis lies within the framework of particle physics, the study of the fundamental building blocks of our universe and their interactions. The findings of these studies are mathematically described using Quantum Field Theory, in which particles are represented as an excitation in a quantised field, and their interactions are mediated by force-carrying particles called bosons. The hugely popular Standard Model (SM) is the collective name given to the accepted and rigorously tested theories that successfully describe three of the four fundamental forces, the electromagnetic, strong and weak forces. At this time Gravity remains absent from the Standard Model. A quick overview is given in this chapter of the formalism of the SM, along with the motivation for physics Beyond the Standard Model (BSM).  Many of the shortcomings of the SM can be explained by the popular theory of Supersymmetry, which provides a framework for the analysis presented in this thesis. 


\section{The Standard Model}

The Standard Model (SM) is a quantum field theory which successfully describes the known elementary particles and their fundamental interactions with respect to the strong, weak and electromagnetic forces. It is a specific type of field theory called a gauge theory, as it describes symmetry under gauge transformations, which are those on degrees of freedom in the system which leave physical results unchanged. Interactions between particles through forces are mediated by exchange of force-carrier particles, which we call \textit{gauge bosons}. 

It is formulated as a non-Abelian Yang-Mills type gauge field theory based on the symmetry group $SU(3)_{C} \times SU(2)_{L} \times U(1)_{Y}$. The strong interactions described by Quantum Chromodynamics (QCD) are represented by $SU(3)_{C}$, and the electromagnetic and weak interactions are represented together due to Electroweak Unification by the group $SU(2)_L \times U(1)_{Y}$. As of yet, the fourth fundamental force Gravity is not included in the Standard Model, but this is seen as of little consequence as gravitational forces are thought to have comparatively little effect on fundamental particles. The current understanding of the theory has been rigorously tested  



 The particle content exists in two types, the bosons which carry forces as described above, and the fermions, which are the building blocks for matter. These two sets are distinguished by the property of Spin, the measure of intrinsic angular moment, as the gauge bosons have spin-1 and the fermions spin-1/2
The fermions which make up all visible matter can be described in three families, or "generations", shown in Equation \ref{eqn:threefams}. Within each generation, there are two sets of particles, those on the left are the leptons, which do not feel the strong force, and those on the right are the quarks, which do. In each generation, there are two quarks, which differ by electoral charge - one has +2/3 and the other -1/3 (in units of the electron charge \textit{e}), a lepton with charge -1 and neutral lepton called a neutrino which is either massless or very light. The three families then are organised in ascending order of mass. The first generation is therefore stable and all ordinary matter is constructed from it, whilst the second and third are liable to decay into particles of the first generation. In addition to each particle detailed here there exists a corresponding antiparticle due to a symmetry in charge and quantum numbers.  

\begin{equation}
\begin{bmatrix}
\nu_{e} & u \\
e & d \\
\end{bmatrix},
\begin{bmatrix}
\nu_{\mu} & c \\
\mu & s \\
\end{bmatrix},
\begin{bmatrix}
\nu_{\tau} & t \\
\tau & b\\
\end{bmatrix}
\label{eqn:threefams}
\end{equation}

\subsection{Spin}
\subsection{Gauge Bosons/INvariance}
\subsection{EWK Unification}
\subsection{Full Particle Content} 


\section{Motivation for New Physics}
	
\subsection{The Hierarchy Problem}
\subsection{Dark Matter}
\subsection{Grand Unification Theories}

\section{Supersymmetry}

\subsection{R-Parity}
\subsection{CMSSM}
\subsection{Current Limits on the CMSSM}
\subsection{Production Mechanisms at the LHC}



